

\chapter{RPM Light Collection}
\label{chap:LightTransport}
\section{RPM Light Collection Performance}
\label{sec:RPMLCP}



\subsection{Proposed Work}
It is proposed to simulate the neutronics, energy deposition, and light transport and collection of the optical photons with the GEANT4 toolkit to ensure that the proposed detector designs will be feasible. 
This will then have two parts:
\begin{enumerate}
  \item the fabrication and simulation of detectors whose performance can be measured in the laboratory, and
  \item the simulation of a RPM in the geometry specified by the neutronics calculations.
\end{enumerate}
It is also proposed to explore the design space of the light collection instrumentation, where such design options include
\begin{itemize}
  \item the use of a wavelength shifter (WLS) to increase the detected optical photons by decrease the optical photons absorbed in the light guide,
  \item the design of a light guide to enhance the photons reflected towards the PMT,
  \item the application of different light reflection techniques (air gap, Teflon tape, aluminized mylar) to enhance the light collection, and
  \item the placement of PMT's for their effective utilization.
\end{itemize}
The developed GEANT4 application will then have the ability to fully simulate the performance of a replacement RPM.

\subsection{Methodology}
The full simulation of the light transport of the RPM will be completed in with the GEANT4 toolkit.
GEANT4 offers three example simulations for generation and tracking of optical photons (\verb+ExampleN06+, \verb+extended/LXe+, and \verb+extended/WLS+).
These examples will be adapted for the purposes of this simulation.
However, it should be noted that the \verb+LXe+ and \verb+WLS+ utilize a sensitive detector which kills tracks, and for the purposes of the simulation of the RPM this will not be implemented.

The validation of the GEANT4 simulations will be completed with the fabrication of a 4 inches by 6 inches layered detector to allow the toolset to be benchmarked against measurements.
Detectors of various materials (boron loaded plastic, lithium loaded polymers, and LiF/ZnS(Ag)) will be tested to avoid a basis in the simulation parameters.
Where possible, the optical parameters of the materials will be taken from the literature.
However, it they cannot be found a similar material for which the property is know will be substituted.
In addition, rather than providing reflectivity parameters for each material, a model of optical surface will be used according to the work of Janecek\cite{5485130}. 

The GEANT4 light transport simulation code base will be written such that bit-string geometries from genetic algorithm optimization can be directly implemented.
With this formulation it will be possible to reuse the genetic algorithm optimization code base, while additionally providing a validation of the GEANT4 neutronic calculations against the MCNPX.
