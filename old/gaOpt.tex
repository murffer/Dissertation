%%%%%%%%%%%%%%%%%%%%%%%%%%%%%%%%%%%%%%%%%%%%%%%%%%%%%%%%%%%%%%%%%%%%%%%%%%
%                                                      									 %
%                  		  	RPM Neutronic Performance   		            	 %
% 									                                                     %
%%%%%%%%%%%%%%%%%%%%%%%%%%%%%%%%%%%%%%%%%%%%%%%%%%%%%%%%%%%%%%%%%%%%%%%%%%
\chapter{Neutronic Optimization}
\label{chap:GARPMOpt}
\section{Introduction}

In this particular problem the placement of a film greatly changes the solution because of the large change in the neutron flux, and traditional gradient based solvers would tend to be trapped in these local minima.
Genetic algorithms were chosen as the optimization technique because they are less susceptible to local minima than other search techniques and tend to preform very well on combinatorial problems such as this one \cite{Mitchell_1997}.



XSDRN will be used to preform an initial parameter study and determine a subset of optimal geometries on which to preform more accurate Monte Carlo (MCNPX) modeling. 
A subset of the highest preforming geometries validated by the detailed model with then be used to test the sensitivity by adjusting the position of the films by fractional amounts.



%%%%%%%%%%%%%%%%%%%%%%%%%%%%%%%%%%%%%%%%%%%%%%%%%%%%%%%%%%%%%%%%%%%%%%%%%%%
%                                                                         %
%                             MCNPX Model                                 %
%                                                                         %
%%%%%%%%%%%%%%%%%%%%%%%%%%%%%%%%%%%%%%%%%%%%%%%%%%%%%%%%%%%%%%%%%%%%%%%%%%%
\section{Optimal Detector Geometries}

\subsecton{10 Length Genomes}
The comparison between the MCNPX simulation and the XSDRN is shown for some of the samples in \autoref{tab:10GenomeXSDRNMCNPXCompare}.
The change in rank is computed by rank of the MCNPX model versus the rank of the XSDRN model.
\begin{table}
  \caption[10 Genome Length RPM Model]{10 Genome Length RPM Model Interactions rates}
  \label{tab:10GenomeXSDRNMCNPXCompare}
  \begin{tabular}{c c | c c | c}
    \toprule
    XSDRN Model & Activity & MCNPX Model & Interaction Rate & Rank Change \\
    \midrule
0111000000 & 10.95 & 0101010000 &  3.08 & \downarray 10
0110100000 & 10.50 & 0101100000 &  2.93 & \uparrow 2
0110010000 & 10.21 & 0110010000 &  2.89 & 0
0101100000 & 10.12 & 0110100000 &  2.85 & \downarray 2
0111100000 & 13.16 & 0110001000 &  2.84 & \downarray 4
    \bottomrule
  \end{tabular}
\end{table}
OUT_0110001000_2_0.635 ->  6.63
OUT_0110010000_2_0.635 ->  6.63
OUT_0101010000_3_0.635 ->  6.54
OUT_0110100000_2_0.635 ->  6.38
OUT_0101010000_1_0.635 ->  6.35

\subsection{20 Lenght Genomes}
Using the XSDRN Model
01111000000000000000 -> 30.39
01111100000000000000 -> 23.93
01011010010000000000 -> 23.46
01111010010000000000 -> 27.14
01111000000000100000 -> 20.22
Using the MCNPX Model
00110010000000000000 ->  3.14
01010010000000000000 ->  3.14
01011000000000000000 ->  2.65
01101000000000000000 ->  2.60
01011010010000000000 ->  4.29
Using the MCNPX Model
OUT_01010010000000000000_3_0.318 ->  6.05
OUT_00110010000000000000_3_0.318 ->  5.90
OUT_01010010000000000000_1_0.318 ->  5.66
OUT_00110010000000000000_3_-0.318 ->  5.57
OUT_00110010000000000000_2_0.318 ->  5.55

\subsection{30 Length Genomes}
Using the MCNPX Model
000100100100000000000000000000 ->  3.39
000101001000000000000000000000 ->  3.27
001001001000000000000000000000 ->  3.19
000110001000000000000000000000 ->  3.14
010001001000000000000000000000 ->  3.13
Using the MCNPX Model
OUT_010001MMMM001M0_5_0.212 ->  5.95
OUT_0001001001MMMMM_6_0.212 ->  5.78
OUT_0001001001MMMMM_3_0.212 ->  5.71
OUT_01000101MMMMM00_5_0.212 ->  5.70
OUT_0001001001MMMMM_3_-0.212 ->  5.61
Using the XSDRN Model
010001010000000000000000000000 -> 13.91
010001010000000010000000000000 -> 16.68
010001010100000000000000000100 -> 17.26
010001000000000000000000100000 ->  9.78
000100100100000000000000000000 ->  7.17

It is observed that the XSDRN model appears to prefer to cluster results.


\section{Conclusions}


