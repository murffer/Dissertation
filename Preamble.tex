%%%%%%%%%%%%%%%%%%%%%%%%%%%%%%%%%%%%%%%%%%%%%%%%%%%%%%%%%%%%%%%%%%%%%%%%%%%
%                                                                         %
%                                 PREAMBLE                                %
%                                                                         %
%%%%%%%%%%%%%%%%%%%%%%%%%%%%%%%%%%%%%%%%%%%%%%%%%%%%%%%%%%%%%%%%%%%%%%%%%%%

%% LANGUAGE and ENDCODING
\usepackage[english]{babel}
\usepackage{lipsum}
\usepackage[utf8]{inputenc}
\usepackage[T1]{fontenc}

%% PACKAGES
\usepackage[toctitles]{titlesec}
\usepackage[page,header]{appendix}
\usepackage[section]{placeins}
\usepackage[usenames,dvipsnames]{xcolor}
\usepackage{microtype}
\usepackage[obeyDraft]{todonotes}
\usepackage{fancyvrb}
\VerbatimFootnotes
\usepackage{algorithmic}
\usepackage{booktabs}
\usepackage{tabularx}
\usepackage{multirow}
\usepackage{longtable}

%% GRAPHICS RELATED
\usepackage{graphicx}
\usepackage[outdir=./tmp/]{epstopdf}
\DeclareGraphicsExtensions{.eps, .pdf, .jpeg, .png}
%% Alt-text for the images (the filename is the alt-text)
%\usepackage{pdfcomment}
%\LetLtxMacro\latexincludegraphics\includegraphics
%\renewcommand{\includegraphics}[2][]{\pdftooltip{\latexincludegraphics[#1]{#2}}{#2}}

%% ALGORTHIM
\usepackage[chapter]{algorithm}
\usepackage{algorithmic}

%% CPATION SETUP
\usepackage{float}
\usepackage{caption}
\usepackage{subcaption}
\captionsetup{belowskip=12pt,aboveskip=4pt}

%% UNITS,  EQUATIONS, and CHEMISTRY
\usepackage{textcomp}
\usepackage{siunitx}
\usepackage[version=3]{mhchem}
\usepackage{mathrsfs}

%% NOMENCLATURE
\usepackage[refpage]{nomencl}  % refer to the page where notation appears
\newcommand{\definevar}[2]{#1 is the #2\nomenclature{#1}{#2}}
\newcommand{\nom}[2]{#1 #2\nomenclature{#1}{#2}}
\renewcommand{\nomname}{List of Notations}
\renewcommand*{\pagedeclaration}[1]{\unskip\dotfill\hyperpage{#1}}
\makenomenclature
%%%%%%%%%%%%%%%%%%%%%%%%%%%%%%%%%%%%%%%%%%%%%%%%%%%%%%%%%%%%%%%%%%%%%%%%%%%
%                                                                         %
%                             Listing Setup                               %
%                                                                         %
%%%%%%%%%%%%%%%%%%%%%%%%%%%%%%%%%%%%%%%%%%%%%%%%%%%%%%%%%%%%%%%%%%%%%%%%%%%
\usepackage{listings}
\lstset{ %
    language=C++,
    basicstyle=\footnotesize\ttfamily,
    numbers=left,
    numberstyle=\tiny\color{gray},
    stepnumber=2,
    numbersep=5pt,
    backgroundcolor=\color{white},
    showspaces=false,
    showstringspaces=false,
    showtabs=false,
    frame=single,
    rulecolor=\color{black},
    tabsize=2,
    breaklines=true,
    breakatwhitespace=false,
    title=\lstname,
    keywordstyle=\color{blue},
    commentstyle=\color{OliveGreen},
    stringstyle=\color{orange}
}
\DeclareCaptionFont{white}{\color{white}}
\DeclareCaptionFormat{listing}{\colorbox[cmyk]{0.43, 0.35, 0.35, 0.01}{\parbox{\dimexpr\textwidth-2\fboxsep\relax}{#1#2#3}}}
\captionsetup[lstlisting]{format=listing,labelfont=white,textfont=white,singlelinecheck=false,margin=0pt,font={bf,footnotesize}}

%% USER COMMANDS
\usepackage{isotope}
\newcommand{\iso}{\isotope}
\newcommand{\figurewidth}{\textwidth}
\DeclareSIUnit\roetgen{R}
\DeclareSIUnit\eV{\electronVolt}
\DeclareSIUnit\count{count}
\DeclareSIUnit\rem{rem}
\DeclareSIUnit\in{in}
\DeclareSIUnit\cps{\count\per\second}
\DeclareSIUnit\cpsngcf{\count\per\second\per\nano\gram\iso[252]{Cf}}

%% Table of Contents
\addto\captionsenglish{%
 \renewcommand{\contentsname}{Table of Contents}%
 \renewcommand{\bibname}{}%
}
