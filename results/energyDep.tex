%%%%%%%%%%%%%%%%%%%%%%%%%%%%%%%%%%%%%%%%%%%%%%%%%%%%%%%%%%%%%%%%%%%%%%%%%%%
%                                                                         %
%                     ENERGY DEPOSITION SIMULATIONS                       %
%                                                                         %
%%%%%%%%%%%%%%%%%%%%%%%%%%%%%%%%%%%%%%%%%%%%%%%%%%%%%%%%%%%%%%%%%%%%%%%%%%%

The average energy deposited by a neutron and gamma reaction was investigated for different film thickness to determine if an optimal thickness exits in a geometry similar to \autoref{fig:EDepSimGeo}.
\autoref{fig:SimEDep} shows that the energy deposition of the gamma quickly falls off as the films get thinner, while it isn't until the neutron films become on the order of the range of the triton that the energy deposition is impacted.
\begin{figure}
  \centering
  \includegraphics[width=\textwidth]{SimulatedEnergyDeposition}
  \caption[Simulated Energy Deposition and Film Thickness]{Simulated energy deposition and film thickness. As the films get thinner it is very unlikely for gamma events (and their secondary electrons) to deposition all of their energy, while above \SI{50}{\um} there is very little impact on the energy deposited by neutrons.\EnergyDepSimGeo}
  \label{fig:SimEDep}
\end{figure}

Interactions that occur near the edge of the film may have a large impact in the energy absorbed in the film for very thin films where a large fraction of the volume is within a mean free path of the surface of the detector material.
For example, for a \SI{25}{\um} film the range of the triton exceeds the thickness of the film.
The energy loss of interactions occurring near the edge of the film was investigated by simulating the energy loss for a large planar detector where the beam spot is \SI{3}{\mm} and the area of the detector face is \SI{10}{\cm}. 
The geometry for this simulation shown in \autoref{fig:EDepSimGeo}, with the interaction position defined to be distance to the first interaction of the beam in the material along the direction of the beam.
\autoref{fig:EDepPosSim} examines the impact of where the interaction took place in the film on the energy deposition.
It is observed that for neutrons events that take place within the center of the films tend to deposit a large majority of their energy in the film, while events that occur on the edge of the film have partial energy depositions in accordance with the ranges of the charged particles.
The Compton edge is observed at \SI{1}{\MeV} in the simulated photon energy deposition for the \SI{1}{\cm} film, as expected.
A secondary effect in having a backing material is also observed for photons in which there is a heightened energy deposition for interactions that occur in the film but near the boundary and electrons are back scattered into the detector material.
\begin{figure*}[ht]
	\centering
	\begin{subfigure}[b]{0.45\textwidth}
    		\includegraphics[width=\textwidth]{{posEDepCo600.025}.png}
		\caption{ \SI{25}{\um} Gamma (\iso[60]{Co})}
	\end{subfigure}%
	~
	\begin{subfigure}[b]{0.45\textwidth}
    		\includegraphics[width=\textwidth]{{posEDepCo6010.0}.png}
		  \caption{ \SI{1}{\cm} Gamma (\iso[60]{Co})}
	\end{subfigure}%
	
  \begin{subfigure}[b]{0.45\textwidth}
    		\includegraphics[width=\textwidth]{{posEDepneutron0.025}.png}
		\caption{ \SI{25}{\um} Neutron}
	\end{subfigure}%
	~
	\begin{subfigure}[b]{0.45\textwidth}
    		\includegraphics[width=\textwidth]{{posEDepneutron10.0}.png}
		  \caption{ \SI{1}{\cm} Neutron}
	\end{subfigure}%
	\caption[Simulated Energy Deposition and Position]{Simulated average energy depositions and the position of the first interactions. The beam is considered to be incident on position 0, and thus interactions that occur on the front of the film have a much higher probability depositing all of their energy. Events that occur on the edge of the film much less likely to deposit all of their available energy.}
	\label{fig:EDepPosSim}
\end{figure*}

\autoref{fig:simKinE} illustrates the simulated kinetic energy of secondary electrons from Compton scattering and from alpha and triton interactions.
It is observed that kinetic energy of the secondary electrons from the neutron reaction products have predominately energies in the kilo-volt range, while the Compton scattering electrons have energies in hundreds of kilo-volts range. 
However, it should be noted that there is only one secondary electron from a Compton scattering and multiple secondary electrons from the reaction products.
The kinetic energy distribution is broken down by the two reaction products in \autoref{fig:SecElecKinEDist}, while the relative number of secondary electrons is shown in \autoref{fig:ReacProdDist}.
It is apparent that the triton contributes more secondary electrons than heavier alpha, while they have similar energies of the secondary electrons.
\begin{figure}[ht]
    \centering
    \includegraphics[width=\textwidth]{NGSecElecKinEDist}
    \caption[Kinetic Energy of Primary Secondary Electron from Compton Scattering and Neutron Reaction Products]{The kinetic energy of the first secondary electron from Compton scattering with \iso[60]{Co} and from \iso[6]{Li} reaction products. The energy distribution of all of the electrons produced in the interactions is shown in \autoref{fig:ERangeAndDist}.\EnergyDepSimGeo}
    \label{fig:simKinE}
\end{figure}
\begin{figure}
 	\centering
  	\includegraphics[width=\textwidth]{AlphaTritonSecElecKinEDist}
	\caption[Kinetic Energy Distribution of Primary Secondary Electrons from the Neutron Reaction Products]{Kinetic energy distribution of the first secondary of the neutron reaction products (alpha and triton) from a \iso[6]{Li} interaction.  Most of the electrons have kinetic energies in the \SI{1}{\keV} range. \EnergyDepSimGeo}
	\label{fig:SecElecKinEDist}
\end{figure}
\begin{figure}
 	\centering
  	\includegraphics[width=\textwidth]{NeutronNumSecElec}
	\caption[Distribution of the Number of Secondary Electrons Produced Per Neutron Interaction]{Distribution of the Number of Secondary Electrons Produced Per Neutron Interaction. The alpha particle produces almost a factor of 16 less photons than the triton. \EnergyDepSimGeo}
	\label{fig:ReacProdDist}
\end{figure}

The average energy deposited was computed for each thickness and normalized by the incident energy for gammas by the Q-value of the reaction for neutrons, and is presented in \autoref{tab:FractionEDep}.
For thickness greater than \SI{150}{\um} there is little benefit in increasing the thickness of the film in terms of energy deposition by neutrons, since over 90\% of the energy is being deposited in the film.
\begin{table}
    \caption[Fractional Energy Deposition per Interaction for Various Thickness]{Fraction of total energy deposited per interaction of a neutron and the photons from \iso[60]{Co} in films of various thickness. The total energy deposited in a neutron event is \SI{4.78}{\MeV}, while the maximum energy deposited from a \iso[60]{Co} is \SI{1.33}{\MeV}.\EnergyDepSimGeo}
      \label{tab:FractionEDep}
	\centering
	\begin{tabular}{c | c c}
\toprule
	Thickness & Gamma Fraction & Neutron Fraction \\
\midrule
	\SI{15}{\um} & 0.010 & 0.531 \\
	\SI{25}{\um} & 0.013 & 0.634 \\
	\SI{50}{\um} & 0.017 & 0.782 \\
	\SI{150}{\um} & 0.032 & 0.927 \\
	\SI{300}{\um} & 0.052 & 0.964 \\
	\SI{600}{\um} & 0.087 & 0.982 \\
	\SI{1}{\mm} & 0.130 & 0.989 \\
	\SI{1}{\cm} & 0.425 & 0.998 \\
\bottomrule
	\end{tabular}

\end{table}

%%%%%%%%%%%%%%%%%%%%%%%%%%%%%%%%%%%%%%%%%%%%%%%%%%%%%%%%%%%%%%%%%%%%%%%%%%%
%                                                                         %
%                  Light Yield and Energy Deposition                      %
%                                                                         %
%%%%%%%%%%%%%%%%%%%%%%%%%%%%%%%%%%%%%%%%%%%%%%%%%%%%%%%%%%%%%%%%%%%%%%%%%%%
\section{Light Yield and Energy Deposition}
The energy deposition and light yield were also investigated by simulations in the GEANT4 environment for polystyrene based films.
These simulations (summarized in \autoref{fig:EDepLightYield}) show that as expected the light output was linear with the energy deposition.
The importance of the particles causing the scintillation events is due to the quenching of the light from heavy charged particles.
Thus, while the gammas from \iso[60]{Co} deposit  less energy than neutrons in a the same thickness of films, the light output is much higher per unit energy deposited.
\begin{figure}
 	\centering
  	\includegraphics[width=\textwidth]{EDepLightYield_Gamma}
		\caption[Energy Deposition and Light Yield in Polystyrene from Co-60 Photons]{Simulated energy deposition from gamma (\iso[60]{Co}) and the corresponding simulated light yield. \SimEDeLYGeo}
\end{figure}
\begin{figure}
 	\centering
  	\includegraphics[width=\textwidth]{EDepLightYield_Neutron}
  	\caption[Energy Deposition and Light Yield in Polystyrene from Neutrons]{Simulated energy deposition and light yield from neutron interactions.  \SimEDeLYGeo}
  \label{fig:EDepLightYield}
\end{figure}
However, it is instructive to look at the distributions of how many photons were created per event.
As the films become thicker and more of the triton energy is captured the response of the triton starts to dominate the alpha (\autoref{fig:NeutronPhotonsGenSim}), resulting in the number of photons peaking around 650 photons for this simulated sample.
For photons, shown in \autoref{fig:GammaPhotonsGenSim}, it is observed that the distribution is flat for very thick films, but for thinner films the probability is greatly increased for an event to generate a low number of photons.
\begin{figure}
  \centering
  \includegraphics[width=\textwidth]{Neutron_PhotonsGenerated_Sim}
  \caption[Number of photons generated from neutron interactions]{Simulated number of photons generated from neutron interactions.  For the \SI{10}{\um} film it is observed that the majority of the photons are generated by a partial energy deposition corresponding to the alpha particle, and this effect tappers off as the films get thicker. \SimEDeLYGeo}
  \label{fig:NeutronPhotonsGenSim}
\end{figure}
\begin{figure}
  \centering
  \includegraphics[width=\textwidth]{Gamma_PhotonsGenerated_Sim}
  \caption[Number of photons generated from gamma interactions]{Simulated number of photons generated from gamma interactions. Thinner films produce distributions that are skewed towards the left due to having less energy deposition. \SimEDeLYGeo}
  \label{fig:GammaPhotonsGenSim}
\end{figure}
