A single film does not have the necessary interactions to fulfill the neutron count rate criteria multiple films are necessary, and the arrangement of these films provides a design space for a replacement RPM.
In the case of the RPM, there are several design parameters that can be explored:
\begin{itemize}
  \item the neutron absorber loading of the film,
  \item the thickness of the film,
  \item the geometry of the film (cylinders or sheets), and
  \item the placement of the films.
\end{itemize}
It is expected that the loading of the film will be limited by the optical clarity, and that the thickness of the film will be determined by the optimization of the energy deposition.
Thus, of the above design parameters only the geometric placement of the films is an available optimization space.

Preliminary work by this author provided a simple design in which the detector layers are linearly placed throughout the detector volume in an alternating fashion.
The analysis of the neutron flux throughout this detector lead to a flat flux profile as shown in \autoref{fig:AltLayerThermalNeutronFraction}.
\begin{figure}
  \includegraphics[width=\textwidth]{ThermalFluxRatioAltLayers}
	\caption{Fraction of the neutron flux that is thermalized through a alternating detector and moderator layered RPM.  The low thermal fluxes result in a poor utilization of the high thermal cross section of \iso[6]{Li}.}
	\label{fig:AltLayerThermalNeutronFraction}
\end{figure}
Effective utilization of the neutron flux is necessary for minimizing the amount of neutron absorber (\iso[6]{Li}) that is used in the detector.
Several different strategies can then be used to optimize the geometry to ensure effective utilization of the thermal cross section of the absorber material.

\section{Optimal Detector Geometries}

\subsection{XSDRN and MCNPX Model Comparison}
The comparison between the MCNPX simulation and the XSDRN is shown for some of the samples in \autoref{tab:10GenomeXSDRNMCNPXCompare} and \autoref{tab:20GenomeXSDRNMCNPXCompare}, where the change in rank is computed by rank of the MCNPX model versus the rank of the XSDRN model.
It is observed that the XSDRN model preformed fairly closely to the MCNPX model, but tended to over predict and favor geometries that had repeated layers and clusters.
\begin{table}
  \caption[10 Genome Length RPM Model]{10 Genome Length RPM Model Interactions rates}
  \label{tab:10GenomeXSDRNMCNPXCompare}
  \begin{tabular}{c c | c c | c}
    \toprule
    Genome & Activity & Interaction Rate & Rank Change \\
    \midrule
  0011010000& 9.30 &  3.82 & $\downarrow$ 13 \\
  0110100000 & 10.50  &  3.81 & 0 \\
  0101010000 & 10.12  & 3.79 & $\downarrow$ 7 \\
 0101100000 &  & 3.79 & $\downarrow$ 1\\
  0011100000 & 9.63 &  3.77 & $\downarrow$ 3 \\
    \bottomrule
  \end{tabular}
\end{table}
\begin{table}
  \caption[20 Genome Length RPM Model]{20 Genome Length RPM Model Interactions rates}
  \label{tab:20GenomeXSDRNMCNPXCompare}
  \begin{tabular}{c c | c c | c}
    \toprule
    Genome & Activity  & Interaction Rate & Rank Change \\
    \midrule
  00100101000000000000 & 7.77 & 3.79 & $\downarrow$ 19 \\
  00011000100000000000&  & 3.78 &  \\
  00011000010000000000& &  3.76 &  \\
  00110001000000000000 &  3.69 & $\downarrow$ 15\\
  01011010010000000000 & 23.46 & 3.66 & $\uparrow$ 1\\
    \bottomrule
  \end{tabular}
\end{table}

\subsection{Pertubations on the MCNPX Model}
Perturbations on the MCNPX were preformed in order to determine if a minimum in the search function was achieved.
A perturbation on a optimal genome is defined as taking each detector slice and translating it a half slice thickness to the left and the right. 
For example, for a five length genome a perturbation on \verb+010100+ would involve first doubling the genome to  \verb+001000100000+ and then perturbing the slices as  \verb+010000100000+, \verb+000100100000+, \verb+001001000000+, and \verb+001000010000+.
The perturbations on the length length genome increased the interaction rate from 3.81 interactions per second to 3.85 interactions per second for the optimal ten length genome, and from 3.85 interactions per second to 3.87 interactions per second.
Thus, it is determined that the genetic algorithm converged to the optimal solution.

\subsection{Optimal Layered Detector Geometries}
The optimal genomes are listed for 10 15, 20 length, and 30 length genomes for a minimum interaction rate of 2.5 interactions per second in \autoref{tab:GAOptRXNRate_25}, 5.0 interactions per neutron per second in \autoref{tab:GAOptRXnRate_5}, and for 7.5 interactions per second in \autoref{tab:GAOptRXNRate_75}.
It is observed that higher length genomes tended to shows a slight decrease in the total interaction rate.


\begin{table}
	\caption[Optimal geometry for 2.5 interactions per second]{Optimal genome geometries for a total minimium interaction rate of 2.5 interactions per second. The detector and simulation is configured per the PNNL criteria.}
	\label{tab:GAOptRXNRate_25}
	\begin{tabular}{m{7cm} m{5cm} m{2cm} }
	\toprule
	Genome & Interaction Rate & Mass \iso[6]{Li} \\
	\midrule
	0011010000 & 3.82 & \SI{12.6}{\gram} \\
	00100101000000000000 & 3.79 &  \SI{12.6}{\gram}  \\
	00010100001000000000000000 & 3.75 &  \SI{12.6}{\gram}  \\
	\bottomrule
	\end{tabular}
\end{table}
\begin{table}
	\caption[Optimal geometry for 5 interactions per second]{Optimal genome geometries for a total minimum interaction rate of 5 interactions per second. The detector and simulation is configured per the PNNL criteria.}
	\label{tab:GAOptRXNRate_5}
	\begin{tabular}{m{7cm} m{5cm} m{2cm} }
	\toprule
	Genome & Interaction Rate  & Mass \iso[6]{Li} \\
	\midrule
	011101001000000 & 5.31 & \SI{21.0}{\gram} \\
	01011010010000000000 & 5.21& \SI{21.0}{\gram} \\
	011001001000010000000000000000 & 5.06 & \SI{21.0}{\gram} \\
	\bottomrule
	\end{tabular}
\end{table}
\begin{table}
	\caption[Optimal geometry for 7.5 interactions per second]{Optimal genome geometries for a total minimium interaction rate of 7.5 interactions per second. The detector and simulation is configured per the PNNL criteria.}
	\label{tab:GAOptRXNRate_75}
	\begin{tabular}{m{7cm} m{5cm} m{2cm} }
	\toprule
	Genome & Interaction Rate & Mass \iso[6]{Li} \\
	\midrule
	01111101110100001000 & 7.56 & \SI{41.2}{\gram} \\
	01111101010010101000 & 7.53 & \SI{41.2}{\gram} \\
	\bottomrule
	\end{tabular}
\end{table}

The neutron flux as it crosses the detector is of interest to examine the utilization of the neutrons.
\autoref{fig:20Length25MinFluxProfile} shows the flux profiles for an optimal geometry for a 20 length genome with a minimum of 2.5 interactions per second and \autoref{fig:20Length5MinFluxProfile} shows the flux profiles for a minimum of 5 interactions per second.
It is observed that the fast flux quickly decreases as the there is a build up of the thermal flux.
Once the thermal flux has reached about \SI{4E-3}{neutrons \per \cm\squared \per\second} it is advantageous to place layers of \iso[6]{Li} to reduce the thermal flux.
A build up of the thermal flux is observed after the last detector layer, and the thermal flux declines as neutrons leave the detector.
\begin{figure}
	\centering
	\includegraphics[width=\textwidth]{20Length25cpsMinFluxProfiles}
	\caption[Neutron Flux Profile for an Optimal 20 Length Genome, minimum 2.5 interactions per second]{Neutron flux profile for a 20 length genome with an interaction rate of 3.82 interactions per neutron. The vertical lines represent detector slices.}
	\label{fig:20Length25MinFluxProfile}
\end{figure}
\begin{figure}
	\centering
	\includegraphics[width=\textwidth]{20Length5cpsMinFluxProfiles}
	\caption[Neutron Flux Profile for an Optimal 20 Length Genome, minimum 5 interactions per second]{Neutron flux profile for a 20 length genome with an interaction rate of 5.31 interactions per neutron. The vertical lines represent detector slices.}
	\label{fig:20Length5MinFluxProfile}
\end{figure}

\subsection{Wrapped Polymer Cylinders}
\label{sec:WrappedCylinders}

In addition to planar detector sheets a possible replacement geometry could be to wrap the detector sheets around a wavelength shifting light core in concentric cylinders and replace the helium tube directly.
MCNPX simulations were completed of geometries containing two, three, and four cylinders of wrapped detector material.
It is envisioned that the detector material could be deposited on a flexible sheet and wrapped to create a cylinder, however, for simplicity a concentric cylinder design is simulated.
The outer diameter of the cylinders were set to be two inches (\SI{2.5}{\cm}) to be a direct replacement of the existing helium three tubes.
The exact placement of the helium tubes in an RPM is not known, so the tubes were placed one-third of the way back in the detector material, and spaced equidistance apart.
The spacing of the tubes inside the radiation portal monitor is described in \autoref{tab:WrappedCylinderPositions}, and shown in \autoref{fig:WrappedCylinderGeo} and \autoref{fig:WrappedCylinderPos}.
\begin{figure}
  \centering
  \includegraphics[width=\textwidth,height=\textheight,keepaspectratio]{WrappedGeoCylinder_Cylinder}
  \caption[Rendering of Wrapped Cylinder Geometry]{MCNPX Rendering of a wrapped cylinder.  There is a \SI{1}{\cm} diameter inner light guide surrounded by films separated by \SI{0.4}{\cm} thick light guides.}
  \label{fig:WrappedCylinderGeo}
\end{figure}
\begin{figure}
  \centering
  \includegraphics[width=\textwidth]{WrappedGeoCylinder_Positions}
  \caption[Positions of Wrapped Cylinders in RPM Cabinet]{MCNPX Rendering of wrapped cylinders placed in an RPM8 Cabinet.}
  \label{fig:WrappedCylinderPos}
\end{figure}
The thickness of the RPM (\SI{12.7}{\cm}) corresponds to the x dimension, where the front of the detector is at x equals \SI{0.0}{\cm}.
The width of the RPM cabinet extends from \SI{-15.25}{\cm} to \SI{15.25}{\cm}, where a y coordinate of zero is on the midline of the detector width.
\begin{table}
  \caption[Wrapped Cylinder Positions]{Positions of the wrapped cylinders in the RPM. The thickness of the RPM corresponds to the x dimension, and the width of the RPM cabinet extends from \SI{-15.25}{\cm} to \SI{15.25}{\cm}}
  \label{tab:WrappedCylinderPositions}
  \begin{tabular}{m{2cm} | m{3cm} m{4cm} }
    \toprule
    Number Cylinders & x coordinate & y coordinate \\
    \midrule
    2 & \SI{4.23}{\cm} & $\pm$ \SI{10.16}{\cm} \\
    3 & \SI{4.23}{\cm} & \SI{0}{\cm}, $\pm$ \SI{7.625}{\cm} \\
    4 & \SI{4.23}{\cm} & $\pm$ \SI{3.05}{\cm}, $\pm$ \SI{9.15}{\cm} \\
    \bottomrule
  \end{tabular}
\end{table}
\begin{table}
  \caption[Two Wrapped Cylinders Interaction Rate]{MCNPX simulated interaction rate of two wrapped cylinders of polymer loaded \iso[6]{LiF} in the RPM8 footprint}
  \label{tab:TwoCylinderResults}
	\begin{tabular}{m{2cm} >{\centering\arraybackslash} m{2cm} >{\centering\arraybackslash} m{2cm} >{\centering\arraybackslash} m{4cm} >{\centering\arraybackslash} m{4cm} }
	\toprule
    Polymer& Fraction \iso[6]{LiF} & Mass \iso[6]{Li}& Interaction Rate  & count rate per mass \\
           &                       &  \centering{\si{\gram}} & \si{\cps\per\ng} \iso[255]{Cf}  & \si{\cps\per\ng \iso[252]{Cf}\per\gram} \\
    \midrule
    PS     &  0.10  & 4.80 &    1.321  $\pm$   0.03 &   0.28\\ 
    PS     &  0.20  &  9.60 &   1.852  $\pm$   0.04 &   0.19\\
    PS     &  0.30  &  14.38 &  2.160 $\pm$   0.04 &   0.15\\
    PEN    &  0.10&  4.77 &   1.325  $\pm$   0.03 &   0.28\\
    PEN    &  0.20&  9.54 &   1.841  $\pm$   0.04 &   0.19\\
    PEN    &  0.30&  14.31 &  2.157 $\pm$   0.04 &   0.15\\ 
    \bottomrule
  \end{tabular}
\end{table}

\begin{table}
  \caption[Three Wrapped Cylinders Interaction Rate]{MCNPX simulated interaction rate of three wrapped cylinders of polymer loaded \iso[6]{LiF} in the RPM8 footprint}
  \label{tab:ThreeCylinderResults}
	\begin{tabular}{m{2cm} >{\centering\arraybackslash} m{2cm} >{\centering\arraybackslash} m{2cm} >{\centering\arraybackslash} m{4cm} >{\centering\arraybackslash} m{4cm} }
	\toprule
    Polymer& Fraction \iso[6]{LiF} & Mass \iso[6]{Li}& Interaction Rate  & count rate per mass \\
           &                       &  \centering{\si{\gram}} & \si{\cps\per\ng} \iso[255]{Cf}  & \si{\cps\per\ng \iso[252]{Cf}\per\gram} \\
    \midrule
    PS     &  0.10  & 7.20&   1.482   $\pm$  0.02 &   0.21\\
    PS     &  0.20  & 14.39&   2.240 $\pm$  0.03 &   0.16\\
    PS     &  0.30  & 21.58&   2.706 $\pm$  0.04 &   0.13\\
   PEN    &  0.10  & 7.15&   1.368  $\pm$  0.02 &   0.19\\
   PEN    &  0.20  &14.31 &  2.119 $\pm$  0.03 &   0.15\\
   PEN    &  0.30  &21.46 &  2.608 $\pm$  0.04 &  0.12 \\
    \bottomrule
  \end{tabular}
\end{table}

\begin{table}
  \caption[Four Wrapped Cylinders Interaction Rate]{MCNPX simulated interaction rate of four wrapped cylinders of polymer loaded \iso[6]{LiF} in the RPM8 footprint}
  \label{tab:FourCylinderResults}
	\begin{tabular}{m{2cm} >{\centering\arraybackslash} m{2cm} >{\centering\arraybackslash} m{2cm} >{\centering\arraybackslash} m{4cm} >{\centering\arraybackslash} m{4cm} }
	\toprule
    Polymer& Fraction \iso[6]{LiF} & Mass \iso[6]{Li}& Interaction Rate  & count rate per mass \\
           &                       &  \centering{\si{\gram}} & \si{\cps\per\ng} \iso[255]{Cf}  & \si{\cps\per\ng \iso[252]{Cf}\per\gram} \\
    \midrule
    PS     &  0.10  &  9.60 &   1.879  $\pm$   0.03 &   0.20\\
    PS     &  0.20  & 19.19 &   2.816 $\pm$   0.04 &   0.15\\
    PS     &  0.30  & 28.77 &   3.360 $\pm$   0.05 &   0.12\\
    PEN    &  0.10  & 9.54&   1.726  $\pm$   0.02 &   0.18\\
    PEN    &  0.20  & 19.08&   2.668 $\pm$   0.04 &   0.14\\
    PEN    &  0.30  &28.62 &   3.234 $\pm$   0.04 &   0.11\\
    \bottomrule
  \end{tabular}
\end{table}

Several pertinent design constraints can be learned from the careful study of these results.
As observed in the 10\% and 20\% loading of \iso[6]{LiF} a doubling of the loading does not imply a doubling of the interaction rate for the same geometry. 
This effect is due to the the flux self-shielding of the outer material layers in which the outer material layer depletes the thermal neutron flux thus making the flux seem harder to the interior layers of the wrapped film assembly.
In addition, it can be observed in all cases the the minimal amount of \iso[6]{Li} has the highest count rate per mass, again due to shelf-shielding.
None of two wrapped cylinder designs spaced by \SI{0.4}{\cm} would be ability to fulfill the radiation portal monitor neutron count rate criteria, thus a direct replacement of the assemblies is not possible.
However,  there would be the possibility (if 78\% of the events are above the lower level discriminator) to use a 30\% loaded PEN or PS film in a four cylinder arrangement as a replacement technology.

To limit the effects of self-shielding the light guide thickness was increased from \SI{0.4}{\cm} to \SI{0.8}{\cm}, and an array of five cylinders (placed at \SI{0}{\cm}, $\pm$\SI{5.08}{\cm} and $\pm$\SI{10.16}{\cm}) were simulated, the geometry of which is shown in \autoref{fig:FiveLayerCylinderGeo}.
The interaction rate for the 30\% loaded polystyrene film was \SI{2.92}{cps\per\ng  \iso[255]{Cf}} (\SI{2.79}{cps\per\ng \iso[255]{Cf} } for the 30\% loaded PEN) using \SI{21.4}{\gram} of \iso[6]{Li}, thus having an interaction rate per mass that is 15\% higher than the four cylinder, 30\% \iso[6]{LiF} loaded PS wrapped cylinder detector.
This indicates that it is more desirable to space out the detector materials than to pack them into a short space.
\begin{figure}
  \centering
  \includegraphics[width=\textwidth,height=\textheight,keepaspectratio]{WrappedGeoCylinder_FiveCylinders}
  \caption[Rendering of Five Layered Cylinders in RPM Cabinet]{MCNPX Rendering of five wrapped cylinders placed in an RPM8 Cabinet. The spacing between the detector layers is \SI{0.8}{\cm}, with the total detector having an interaction rate of  \SI{2.92}{cps\per\ng  \iso[255]{Cf}}.}
  \label{fig:FiveLayerCylinderGeo}
\end{figure}

The five cylinder design (with \SI{0.8}{\cm} spacing between detector film layers) had three layers of detectors in each of the five cylinders.
The inner most layer, with an inner radius of \SI{0.5}{\cm}, contained 13\% of the mass \iso[6]{Li}, while the second layer contained 33\%  and the outer layer (inner radius \SI{2.12}{\cm}) contained the majority of the absorber at 54\%.
The interaction rate of each detector film layer in a cylinder is described for all five cylinders in \autoref{tab:WrappedCylinderRXNRate}.
It is observed that the outer cylinder (located at $\pm$ \SI{10.16}{\cm})reaction rates are very similar to the other three, suggesting that the edges effect and the flux depression due to presences of the other cylinders are negligible.
In addition, it is observed that when the interaction rate is normalized by the fraction that each cylinder occupies in the assembly there is little differnece (beyond the 5\% statistical convergence on the tallies) in the reaction rate per unit mass between the layers, thus suggesting a close to optimal material usage is achieved.
If it was necessary, however, to trim material the inner cylinder would be an ideal candidate due to contributing little in the way of total counts.
In this design the outer cylinders had over 56\% of the total interaction rate, while the innermost cylindrical film layers contributed less than 12\% to the total interaction rate.
\begin{table}
	\caption[Neutron Interactions per Layer in Cylinder]{Simulated Interaction rate (per nanogram \iso[252]{Cf}). The tally convergence was within 5\% for all tally values.}
	\label{tab:WrappedCylinderRXNRate}
	\begin{tabular}{m{4cm} | m{2cm} m{1.75cm} m{1.5cm} m{1.5cm} m{1.5cm}}
		\toprule
		Fraction of Assembly & \SI{-10.16}{\cm} & \SI{-5.08}{\cm} & \SI{0}{\cm} & \SI{5.08}{\cm} & \SI{10.16}{\cm}\\
		\midrule
		0.13 & 0.072 & 0.074 & 0.069 & 0.070 & 0.062 \\
		0.33	& 0.180 & 0.201 & 0.186 & 0.189 & 0.172 \\
		0.54	& 0.318 & 0.351 & 0.330 & 0.338 & 0.310 \\
		\bottomrule
	\end{tabular}
\end{table}
