
%%%%%%%%%%%%%%%%%%%%%%%%%%%%%%%%%%%%%%%%%%%%%%%%%%%%%%%%%%%%%%%%%%%%%%%%%%
%                                                      									 %
%                  		  	RPM Light Collection                           %
% 									                                                     %
%%%%%%%%%%%%%%%%%%%%%%%%%%%%%%%%%%%%%%%%%%%%%%%%%%%%%%%%%%%%%%%%%%%%%%%%%%
\section{RPM Light Collection Performance}
\label{sec:RPMLCP}

There is no assurance that the detectors designed based on interaction rate would be feasible to construct; due to their low light output and opaqueness collecting the light from scintillation events would be extremely difficult.  
Additional simulation work then needs to be completed to ensure that a RPM in the layered detector design has a realistic method of collecting the light emitted from the scintillation events.
Light transport modeling provides a way to calculate the performance of such a design while providing insights for the improvement of a detector design.

Several previous authors have used the GEANT4 toolkit to simulate the light collection efficiency of their detector designs.
In PNNL 14283 the authors looked at a variety of different PMT placement and detectors designs to increase the light output of a detector in the Advanced Large-Area Plastic Scintillators (ALPS) project \cite{pnnl_14283}.
The authors found that for a \SI{127}{\cm} by \SI{57}{\cm} by \SI{5}{\cm} slab of BC-408 wrapped in a loose foil of 85\% reflectivity that the light output could be almost doubled by doubling the number of PMT's.
These results are summarized in \autoref{tab:PNNLLightCollectionEfficiency}.
\begin{table}
  \centering
  \caption[PNNL Light Collection Efficiencies]{Light collection efficiencies of several detector designs simulated by PNNL\cite{pnnl_14283}.}
  \label{tab:PNNLLightCollectionEfficiency}
  \begin{tabular}{c|c c}
  \toprule
  & \multicolumn{2}{c}{Light Collection Efficiency} \\
  Number of PMTs  & 2-in PMT & 5-in PMT \\
  \midrule
  2 & 7.0\% & 18.8\% \\
  4 & 13.3\% & 30.7\ \\
  6 & 18.4\% & 40.2\% \\
  \bottomrule
  \end{tabular}
\end{table}
In addition, other authors have reported on the simulation performance of a light guides and photon attenuation using the GEANT4 toolkit.
An overview of the light transport and scintillation processes available in GEANT4 is presented in \cite{riggi_introducing_2011}.
The authors also simulated a \SI{1}{\m} by \SI{1}{\cm} by \SI{1}{\cm} plastic strip in order to calculate the photon attenuation in the plastic.
Two orders of magnitude drop in the number of photons was calculated for photons collected \SI{90}{\cm} from the distance of emission, and was highly dependent of the reflectivity of the material encasing the plastic strip\cite{riggi_introducing_2011}.
Polymeric detectors containing PPO/POPOP as the fluor have also been modeled for the light transport with GEANT4\cite{5485130}.
This provides a measure of confidence that the films fabricated at the University of Tennessee containing PPO/POPOP can also be simulated without undo burden in finding light transport properties.

\subsection{Proposed Work}
It is proposed to simulate the neutronics, energy deposition, and light transport and collection of the optical photons with the GEANT4 toolkit to ensure that the proposed detector designs will be feasible. 
This will then have two parts:
\begin{enumerate}
  \item the fabrication and simulation of detectors whose performance can be measured in the laboratory, and
  \item the simulation of a RPM in the geometry specified by the neutronics calculations.
\end{enumerate}
It is also proposed to explore the design space of the light collection instrumentation, where such design options include
\begin{itemize}
  \item the use of a wavelength shifter (WLS) to increase the detected optical photons by decrease the optical photons absorbed in the light guide,
  \item the design of a light guide to enhance the photons reflected towards the PMT,
  \item the application of different light reflection techniques (air gap, Teflon tape, aluminized mylar) to enhance the light collection, and
  \item the placement of PMT's for their effective utilization.
\end{itemize}
The developed GEANT4 application will then have the ability to fully simulate the performance of a replacement RPM.

\subsection{Methodology}
The full simulation of the light transport of the RPM will be completed in with the GEANT4 toolkit.
GEANT4 offers three example simulations for generation and tracking of optical photons (\verb+ExampleN06+, \verb+extended/LXe+, and \verb+extended/WLS+).
These examples will be adapted for the purposes of this simulation.
However, it should be noted that the \verb+LXe+ and \verb+WLS+ utilize a sensitive detector which kills tracks, and for the purposes of the simulation of the RPM this will not be implemented.

The validation of the GEANT4 simulations will be completed with the fabrication of a 4 inches by 6 inches layered detector to allow the toolset to be benchmarked against measurements.
Detectors of various materials (boron loaded plastic, lithium loaded polymers, and LiF/ZnS(Ag)) will be tested to avoid a basis in the simulation parameters.
Where possible, the optical parameters of the materials will be taken from the literature.
However, it they cannot be found a similar material for which the property is know will be substituted.
In addition, rather than providing reflectivity parameters for each material, a model of optical surface will be used according to the work of Janecek\cite{5485130}. 

The GEANT4 light transport simulation code base will be written such that bit-string geometries from genetic algorithm optimization can be directly implemented.
With this formulation it will be possible to reuse the genetic algorithm optimization code base, while additionally providing a validation of the GEANT4 neutronic calculations against the MCNPX.
