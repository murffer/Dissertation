%%%%%%%%%%%%%%%%%%%%%%%%%%%%%%%%%%%%%%%%%%%%%%%%%%%%%%%%%%%%%%%%%%%%%%%%%%%
%                                                                         %
%                         Introduction to Scintillation                   %
%                                                                         %
%%%%%%%%%%%%%%%%%%%%%%%%%%%%%%%%%%%%%%%%%%%%%%%%%%%%%%%%%%%%%%%%%%%%%%%%%%%
Scintillation detecctors (the detectors on which this work is based) utilize a scintillator to convert ionizing radiation into photons, and then transporting and capturing those the emitted photons, commonly with a photomultiplier tube (PMT).
The electrical signal from the PMT is then an indicator of a radiation event in the detector's scintillator material.
The current RPM's with \iso[3]{He} are ion chamber detectors, however, the proposed \iso[6]{Li} glass detectors and \iso[6]{LiF} doped ZnS:Ag detectors are scintillation based detectors. 

\subsection{Organic Scintillators}
In an organic scintillator the transitions between the valence electrons (either the singlet states or triplet states) of the scintillat molecule emitt photons as they relax to their ground state.
An incoming particle (generally electrons liberated from the energy deposition of the ionizing radiaiton) can either excite a vibrational mode of the scintillant or a valance electron.
If a vibrational mode is excited the scintillitan will relax to the ground state without any emission of light.
An exictation to the first excited singlet state, $S^1$,  
The
The ground state S0 is a singlet state above which are the excited singlet states (S*, S**,…), the lowest triplet state (T0), and its excited levels (T*, T**,…). 
An incoming particle can excite either an electron level or a vibrational level. The singlet excitations immediately decay (< 10 ps) to the S* state without the emission of radiation (internal degradation). The S* state then decays to the ground state S0 (typically to one of the vibrational levels above S0) by emitting a scintillation photon. This is the prompt component or fluorescence. The transparency of the scintillator to the emitted photon is due to the fact that the energy of the photon is less than that required for a S* → S0 transition (the transition is usually being to a vibrational level above S0).[15]
When one of the triplet states gets excited, it immediately decays to the T0 state with no emission of radiation (internal degradation). Since the T0 → S0 transition is very improbable, the T0 state instead decays by interacting with another T0 molecule:[15]

and leaves one of the molecules in the S* state, which then decays to S0 with the release of a scintillation photon. Since the T0-T0 interaction takes time, the scintillation light is delayed: this is the slow or delayed component (corresponding to delayed fluorescence). Sometimes, a direct T0 → S0 transition occurs (also delayed), and corresponds to the phenomenon of phosphorescence (note that the difference between delayed-fluorescence and phosphorescence lies in the difference in the wavelengths of the emitted optical photon in a S* → S0 transition versus a T0 → S0 transition).
Organic scintillators can be dissolved in an organic solvent to form either a liquid or plastic scintillator. The scintillation process is the same as described for organic crystals (above); what differs is the mechanism of energy absorption: energy is first absorbed by the solvent, then passed onto the scintillation solute (the details of the transfer are not clearly understood).[15]

Several factors need to be considered in order of a material to be considred as a scintilaltor:
\begin{itemize}
  \item The light yield should be linearly proportional to the energy deposition
  \item The material should be optically transparent to it's own wavelength of emission
  \item A short rise time with a fast decay time
\end{itemize}
Scintillators convert the energy deposited in the material by ionizing radiation in to photons.


