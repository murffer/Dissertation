%%%%%%%%%%%%%%%%%%%%%%%%%%%%%%%%%%%%%%%%%%%%%%%%%%%%%%%%%%%%%%%%%%%%%%%%%%%
%                                                                         %
%                         Introduction to Scintillation                   %
%                                                                         %
%%%%%%%%%%%%%%%%%%%%%%%%%%%%%%%%%%%%%%%%%%%%%%%%%%%%%%%%%%%%%%%%%%%%%%%%%%%
Scintillation detecctors (the detectors on which this work is based) utilize a scintillator to convert ionizing radiation into photons, and then transporting and capturing those the emitted photons, commonly with a photomultiplier tube (PMT).
The electrical signal from the PMT is then an indicator of a radiation event in the detector's scintillator material.
The current RPM's with \iso[3]{He} are ion chamber detectors, however, the proposed \iso[6]{Li} glass detectors and \iso[6]{LiF} doped ZnS:Ag detectors are scintillation based detectors. 

\subsection{Organic Scintillators}
An organic scintillator generally has a $\pi$-electron structure, as shown in \autoref{fig:pielectron}.
An incoming particle (generally electrons liberated from the energy deposition of the ionizing radiaiton) then excites one of the modes of the $\pi$-electron structure.
Higher singlet states rapidly (on the order of picoseconds) relax to the first singlet state, and excessive vibrational energy (populations of the vibrational sates) is lost.
Thus, after a short period of time the entire excitation population is in the $S_10$ state, and the decay of this state creates the prompt fluorescence.
\begin{figure}
  \centering
  \includegraphis[width=\textwidth]{PiElectronSates}
  \caption[$\pi$ Electron Structure]{Typicall $\pi$-electron structure of an organic molecule. The ground state of the molecule is shown as $S_0$, and excited single states are $S_1$, $S_2$ etc The triplet states are $T_1$, $T_2$, with the vibrational states as $S_OO$, $S_01$, $S_02$ and so forth. Figure from Wikipedia.}
  \label{fig:pielectron}
\end{figure}
The exictations of triplet states typically yield delayed scintillation events or phosphorescence.
An exicted triplet states immediately decays to the $T_0$ state by internal degration without a photon emission.
The $T_0$ state typically decays by interacting with another $T_0$ state in a $T_0 + T_0 \to S^* + S_0 + h\nu$ transition.
The excited singlet state $S^*$ then also decays to the $S_0$ state.
The $T_0 + T_0$ transitions is slower than direct singlet state dexcitations, and results in a slow component of the pulse, which can be used for pulse shape discrimination.

Several factors need to be considered in order of a material to be considred as a scintilaltor:
\begin{itemize}
  \item The light yield should be linearly proportional to the energy deposition
  \item The material should be optically transparent to it's own wavelength of emission
  \item A short rise time with a fast decay time
\end{itemize}
Scintillators convert the energy deposited in the material by ionizing radiation in to photons.


