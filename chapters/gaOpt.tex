\chapter{Neutronic Optimization}
\label{ch:GAOpt}
\section{Introduction}
Effective utilization of the neutron flux and is necessary for minimizing the amount of neutron absorber (\iso[6]{Li}) that is used in the detector.

Several possible geometries have been considered for the films. 
They are generally a vertical or horizontal layer of the films or a vertical or horizontal aragment of the films aranged in cocentric cylindars.
\section{Methods}


\subsubsection{MCNPX Detector Modeling}
\label{sec:MCNPDetectorModelingMethod}
The performance of films is simulated in MCNPX, a Monte Carlo transport code\cite{pelowitz_mcnpx_????}.
The geometry is as in the PNNL reports, namely a nano-gram of \iso[252]{Cf}  encased in \SI{0.5}{\cm} of lead and \SI{2.5}{\cm} of HDPE. 
The size of the RPM8 is \SI{12.7}{\cm} deep, by \SI{30}{\cm} wide and \SI{2}{\m} tall.

The interaction rate is calculated using the a cell flux tally in MCNPX and a tally multiplier card.
The tally multiplier card (FMn) is used to calculated any quantity of the form \eqref{eqn:FMCardForm} \cite{pelowitz_mcnpx_????}
\begin{align}
  \label{eqn:FMCardForm}
  I &= C\int\phi(E)\Re_m(E)dE
\end{align}
where \definevar{$I$}{Interaction rate}, \definevar{$\phi(E)$}{Energy dependent fluence} , \definevar{$\Re_m(E)$}{Response function operator} and $C$ is an arbitrary scalar for normalization.
An general example of the use of the FM card is shown in Listing \ref{lst:GeneralFMExample}, which is taken from the MCNPX manual \cite{pelowitz_mcnpx_????}.
% See pg. 4-41 of the MCNP manual
\begin{lstlisting}[caption={[Example usage of the FM card]Example usage of the FM card to calculate the number of reactions per \si{\cm\cubed} of type R in cell 8 of material M. The normalization is by atomic density, signified by the -1},label={lst:GeneralFMExample}]
F104: N 8
FM104 -1 M R
\end{lstlisting}

The reaction rate $\iso[6]{Li}\left(\text{n},\text{t}\right)\alpha$ can be calculated by then applying the appropriate input for the FMn card and using an F4 card to calculate $\phi(E)$.
It should be noted that depending on the form of the cell flux card it may be necessary to normalize by the volume of the cell, $\forall$.
\nomenclature{$\forall$}{Volume of the cell}

This is shown in Listing \ref{lst:InteractionRateRPM}, where the reaction number is 105 and the material number of the detector is 3.
The interaction rate in a simulated RPM8 replacement detector is calculated in a similar manner as the simulation of the measured detectors; the interaction rate as computed by the \verb+FMn+ is multiplied by the source strength and volume if necessary.
An example of the MCNPX input cards is shown in Listing \ref{lst:InteractionRateRPM}.
Given that there the thermal response is not desired, there is no need to subtract out the differences between the spectra, and the interaction rate is simply \eqref{eqn:RPM8InteractionRate}.
Note that in this calculation the source strength is set to be \SI{1}{\nano\gram} \iso[252]{Cf}, which has a neutron emission rate of \SI{2.3E3}{neutron\per\second}.
This is in accordance with the direct evaluation of the PNNL criteria, which require a absolute neutron count rate of \SI{2.5}{count\per\second\per\nano\gram\iso[252]{Cf}}.
\begin{lstlisting}[caption={[RPM8 ${}^{6}\text{Li}\left(\text{n},\text{t}\right)\alpha$ Reaction Rate]RPM8 ${}^{6}\text{Li}\left(\text{n},\text{t}\right)\alpha$ Reaction Rate. The detector is all of the layers of cell 500 inside universe 610. This tally is multiplied by an SD card to normalize by the volume},label={lst:InteractionRateRPM}]
FC4 (n,t) Reactions in Thin Film (Neutron Detector)
F4:n (500<610)
SD4 1
FM4 -1 3 105
\end{lstlisting}
\begin{align}
  \label{eqn:RPM8InteractionRate}
  I_{\text{sim}} &= S_0 I \\
  &= \SI{2.3E3}{neutron\per\second} I
\end{align}

$I_{\text{sim}}$ provides the total number of simulated neutron interactions in the detector.
However, not all of these interactions will lead to counts above the pulse height discriminator setting necessary for meeting the gamma intrinsic efficiency.
This is corrected for by scaling $I_{\text{sim}}$ by the fraction of counts, $\eta$, that occur above the gamma LLD \eqref{eqn:FractionOfCountsDefination}, \eqref{eqn:RPMCountRate}.
\begin{align}
  \label{eqn:FractionOfCountsDefination}
  \eta \equiv \frac{\int_{MLLD}^\infty p(x)dx}{\int_0^\infty p(x)dx}
\end{align}
\nomenclature{$p(x)$}{Measured spectra, as a function of channel number}
\begin{align}
 \label{eqn:RPMCountRate}
 \text{Count Rate} &= I_{\text{sim}} \eta
\end{align}

\subsubsection{XSDRN Detector Modeling}
A full 3D Monte Carlo treatment of the problem provides an accurate knowledge of the interaction rate at the expense of higher computation times. 
In order to preform a faster search of the parameter space a simple one dimenensional model was desired.
\subsection{Genetic Algorthim Introduction}
The genetic algorithm typically consist of four tasks: creating an initial population, evaluating that populations fitness, selecting members of the current population to breed, and then applying genetic operators to the selected members to breed the new population. 
This is completed for either a maximum generation is reached or the desired fitness is achieved. 


First, however, it is necessary to become familar with the terms used in genetic programming.
\begin{itemize}
  \item \verb+genome+ The set of all genes of an individual, where a gene is a problem parameter.  The problem representation
  \item \verb+population+ All individuals that represent a solution to the problem at a given point in time
  \item \verb+selection+ Choosing which individuals go on to produce new offspring
\end{itemzie}
\subsubsection{Initial Population}

The inital population


\subsection{Population Selection}

Population selection was achieved using a hybrid approach of tournament
selection and rank selection. In rank selection the individuals were
selected based on their rank, where the rank was determined by their
fitness (lower SSE correspond to a higher fitness). Tournament selection
was used to generate the other component of the population for breeding.
\subsection{Genetic Operators}

Individuals selected for reproduction are subjected to genetic operators
to breed the next generation. Genetic programs generally contain two
genetic operators, crossover and mutation. Crossover serves to create
new members of the population by interchanging the genetic material
(tree structure) of two parents in which significant changes in the
tree solutions are achieved. Mutation serves to slightly modify an
existing solution. 


\subsubsection{Crossover}

Crossover is defined in genetic programming as the swapping of sub-trees
between individuals in the population, which is meant to mimic the
crossover of genes that occurs during evolution. As the solutions
are recursively defined, crossover was implemented by selecting randomly
selecting a two nodes from different trees and swapping their parents.


\subsubsection{Mutation}

Mutation, unlike crossover, does not involve changes to the structure
of the solution but rather operators within the solution. For example,
in symbolic regression mutation will randomly (defined by the mutation
rate) change the node function and leaf values, but it will not add
or prune branches to the tree structure. Mutation was implemented
as a recursive traversal of the tree, switching the function types
with other functions types according to a user supplied probability.
The terminals mutations were implemented in a similar fashion, but
if the node was a value leaf the leaf was randomly set to a new random
variable. An example of mutation is shown in Figure \ref{Mutation}.

\subsubsection{Selection Techniques}
Several differnet selection technqiues are available to select the individuals from one population to reproduce in the next.
Among the most common are proportional fitness selection, rank selection, tournament selection, and roulette selection.
Proporitional fitness selection
Rank selection occurs when the individuals are ranked by their fitness
The analogy to a roulette wheel can be envisaged by imagining a roulette wheel in which each candidate solution represents a pocket on the wheel; the size of the pockets are proportionate to the probability of selection of the solution.

\section{Conclusions}

A physical basis of the optimal solution found by the genetic algorthim can be found by observing the form of the optimal solutions.
These solutions involve an intial moderator layer in order to ensure that all of the neutrons are thermalized to (increasing the thermal fraction by \textbf{SOME PRECENT}).
After this moderator layer a film layer is placed to utilize this neutron spectra; however not all of the thermal neutrons are captured (as the mean free path of a neutron in polyethylene is about \si{0.37}{\cm} and thus some pass through the material) and another abosrber layer is needed to capture those neutrons.  
The neutron flux is then moderated again, and additional layers of detectors are needed to capture this neutron cross section.
However, it is desirably to have a large neutron reflector in the portal monitor to reflect neturons back into the detector slices. 
Theortically this reflector should be as large as possible, but the limited space of the RPM provides a constraint.
A parameter study with a single detector slice between a moderator and reflector constrained by the radiaiton portal monitor design showed that it is desriable to have around \textbf{So much} moderator leaving the majority of the RPM to be left for the reflector.
Thus it is demonstrated that a large reflector is desired.
