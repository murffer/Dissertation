\chapter{Summary, Deliverables and Timetable}
\label{ch:SummaryDeliverables}
\section{Summary}
Defense against nuclear terrorism relies upon the ability to accurately detect and identify Special Nuclear Material (SNM).
The current standard for neutron detection in portal monitors is \iso[3]{He}, which is a diminishing resource. 
A significant amount of research is then focused on developing new (or optimizing an existing) neutron detection system to serve as a replacement RPMs.
Previous work by this author focused on the characterization of replacement polymeric thin film detector technologies for RPMs.  
The proposed work will focus on designing a framework for determining an optimized detector design for replacement RPMs based on polymeric thin films.
This work will be novel in its utilization of high fidelity energy deposition to explain the origins of neutron-gamma discrimination through the range of secondary electrons from the neutron and photons reactions in their subsequent energy deposition.
The neutronic performance of a film will then be assessed by calculating the performance of a full scale replacement RPM in the test environment with MCNPX.
Non-linear search techniques will then be used to find the optimal geometry of the MCNPX model of the detector that best utilizes the neutron absorber.
Finally, the scintillation light from the detectors will be simulated with the GEANT4 toolkit, and the performance of the detector based upon the collected light signal reported.
Different light collection strategies will be employed to maximize the light collection efficiency and minimize the cost of electronics necessary for the detector.


\section{Delerivables and Timetable} 
The proposed optimization strategy will first optimize on the microscopic domain of the energy deposition of a single layered detector (supporting the simulations with measurements) and then optimize the performance on the macroscopic domain of a large area plastic scintillator detector.
What has been proposed is essentially an architecture for the optimization that can be broken down into modules.
For each of the subsystems, the developed code will be well documented in order to ensure that the toolkit is portable and with the ability to be extended.
In addition, self-contained repositories of the code base will be made available.
Doxygen, a standard tool for extracting documentation for annotated C++ sources will be used to develop HTML based documentation files and a project manual.
If time permits a tutorial will be developed to assist others in replicating this work.
Measurements made to validate the simulations will also be well documented, and hopefully self contained.
This is to ensure the portability of this work.
\autoref{fig:OptArc} is a flow chart of the proposed optimization process in which the energy deposition simulations influence the inputs of the neutronics modeling and the light transport simulations.
If either of these simulations indicate that the DHS/DNDO detector criteria will not be meet, it is then necessary to modify the detector thickness or material in order to design a detector that will meet the criteria. 
\begin{figure}
  \centering
  \includegraphics[width=0.75\textwidth]{optArch}
  \caption[Optimization Architecture]{Optimization Architecture.  The energy deposition module provides inputs into the MCNP neutronics model and the GEANT4 Light transport simulations. Parameters available for optimization are shown in yellow, while measurement feedback is shown in cyan.}
  \label{fig:OptArc}
\end{figure}

The timelines for the proposed work is presented in the following charts.
\autoref{fig:AdmTimeline} shows a broad overview of the project milestones and accomplishements.
In \autoref{fig:CompletedWorkTimeline} the energy deposition and neutronics optimization by genetic algothrims are shown.
Finally \autoref{fig:LightTransportTimeline} presents the timeline for simulation of light transport.
%%%%%%% ADMINISTRATIVE TIMELINE %%%%%
\begin{sidewaysfigure}
\begin{center}
\begin{ganttchart}[x unit=8.5mm,vgrid,time slot format=isodate,compress calendar,today=2013-08-05,today offset=.5,today label=Current Week,today rule/.style={draw=blue,ultra thick}]{2012-06-12}{2013-12-30}
\gantttitlecalendar{year} \\
\ganttgroup[progress=100]{Energy Modeling}{2012-06-15}{2013-06-01}\\
\ganttgroup[progress=90]{Neutronics Optimization}{2012-12-15}{2013-3-30}\\
\ganttgroup[progress=75]{GEANT4 Light Transport}{2013-4-1}{2013-9-1}\\
\ganttgroup[progress=50]{Write Up and Conclusions}{2013-7-1}{2013-10-1}\\
\ganttmilestone{Dissertation Proposal}{2013-8-9}\\
\ganttbar{Disserrtation Defense}{2013-10-1}{2013-11-1}\\
\ganttmilestone{Dissertation to Trace}{2013-11-27}\\
\ganttmilestone{Pass/Fail Form}{2013-11-27}\\
\end{ganttchart}
\end{center}
\caption{Proposed Adminstrative Timeline}
\label{fig:AdmTimeline}
\end{sidewaysfigure}
%%%%%%% COMPLETED WORK %%%%%%%
\begin{sidewaysfigure}
\begin{center}
\begin{ganttchart}[x unit=8.5mm,vgrid,time slot format=isodate,compress calendar,today=2013-08-05,today offset=.5,today label=Current Week,today rule/.style={draw=blue,ultra thick}]{2012-06-12}{2013-12-30}
\gantttitlecalendar{year} \\
\ganttgroup[progress=100]{Energy Modeling}{2012-06-15}{2013-06-01}\\
\ganttbar[progress=100]{Energy Deposition}{2012-06-15}{2013-4-15} \\
\ganttbar[progress=100]{Range}{2013-5-25}{2013-6-15} \\
\ganttmilestone{IEEE Secondary Electron Paper}{2013-06-15}\\
\ganttgroup[progress=90]{Neutronics Optimization}{2012-12-15}{2013-3-30}\\
\ganttbar[progress=100]{MCNP Neutronics}{2012-12-15}{2013-1-15} \\
\ganttbar[progress=100]{Scripted Input Decks}{2013-1-30}{2013-2-5}\\
\ganttlinkedbar[progress=90]{Genetic Algorthims}{2013-2-15}{2013-3-10}\\
\ganttmilestone{ANS Presentation}{2013-11-10}
\end{ganttchart}
\end{center}
\caption[Timeline of Completed Work]{Timeline of Completed Work.}
\label{fig:CompletedWorkTimeline}
\end{sidewaysfigure}
%%%% LIGHT TRANSPORT TIMELINE
\begin{sidewaysfigure}
\begin{center}
\begin{ganttchart}[x unit=8.5mm,vgrid,time slot format=isodate,compress calendar,today=2013-08-05,today offset=.5,today label=Current Week,today rule/.style={draw=blue,ultra thick}]{2012-06-12}{2013-12-30}
\gantttitlecalendar{year} \\
\ganttgroup[progress=75]{GEANT4 Validation}{2013-1-10}{2013-7-10}\\
\ganttbar[progress=100]{Flir Detectors}{2013-4-15}{2013-4-30}\\
\ganttbar[progress=100]{Fabrication of Detectors}{2013-1-10}{2013-7-10} \\
\ganttlinkedbar[progress=100]{GS20 Simulation}{2013-06-1}{2013-7-10}\\
\ganttlinkedbar[progress=80]{Layred Simulations}{2013-6-15}{2013-7-15}\\
\ganttgroup[progress=75]{GEANT4 RPM Light Transport}{2013-6-1}{2013-10-1}\\
\ganttbar[progress=50]{Prelimiary Simuation}{2013-7-1}{2013-7-20} \\
\ganttlinkedbar[progress=25]{Parameter Studies}{2013-8-1}{2013-9-1}\\
\ganttlinkedbar[progress=0]{Optimization}{2013-9-1}{2013-10-1}\\
\end{ganttchart}
\end{center}
\caption{Light Transport Timeline}
\label{fig:LightTransportTimeline}
\end{sidewaysfigure}
