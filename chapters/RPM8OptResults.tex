\section{Optimal Detector Geometries}

\subsection{XSDRN and MCNPX Model Comparison}
The comparison between the MCNPX simulation and the XSDRN is shown for some of the samples in \autoref{tab:10GenomeXSDRNMCNPXCompare} and \autoref{tab:20GenomeXSDRNMCNPXCompare}, where the change in rank is computed by rank of the MCNPX model versus the rank of the XSDRN model.
It is observed that the XSDRN model preformed faily closesly to the MCNPX model, but tended to over predict and favor geometries that had repeated layers and clusters.
\begin{table}
  \caption[10 Genome Length RPM Model]{10 Genome Length RPM Model Interactions rates}
  \label{tab:10GenomeXSDRNMCNPXCompare}
  \begin{tabular}{c c | c c | c}
    \toprule
    XSDRN Model & Activity & MCNPX Model & Interaction Rate & Rank Change \\
    \midrule
  0111000000 & 10.95 & 0101010000 &  3.08 & $\downarrow$ 10 \\
  0110100000 & 10.50 & 0101100000 &  2.93 & $\uparrow$ 2 \\
  0110010000 & 10.21 & 0110010000 &  2.89 & 0 \\
  0101100000 & 10.12 & 0110100000 &  2.85 & $\downarrow$ 2 \\
  0111100000 & 13.16 & 0110001000 &  2.84 & $\downarrow$ 4 \\
    \bottomrule
  \end{tabular}
\end{table}
\begin{table}
  \caption[20 Genome Length RPM Model]{20 Genome Length RPM Model Interactions rates}
  \label{tab:20GenomeXSDRNMCNPXCompare}
  \begin{tabular}{c c | c c | c}
    \toprule
    XSDRN Model & Activity & MCNPX Model & Interaction Rate & Rank Change \\
    \midrule
  01111000000000000000 & 30.39 & 00110010000000000000 &  3.14 & $\downarrow$ 10 \\
  01111100000000000000 & 23.93 & 01010010000000000000 &  3.14 & $\downarrow$ 7 \\
  01011010010000000000 & 23.46 & 01011000000000000000 &  2.65 & $\downarrow$ \\
  01111010010000000000 & 27.14 & 01101000000000000000 &  2.60 & $\downarrow$ 5\\
  01111000000000100000 & 20.22 & 01011010010000000000 &  4.29 & $\uparrow$ 2\\
    \bottomrule
  \end{tabular}
\end{table}

\subsection{Pertubations on the MCNPX Model}
Pertubations on the MCNPX were preformed inorder to determine if a minimum in the search function was acheived.
For the ten length genomes where each slice is \SI{1.27}{\cm} and a pertubation moves a slice \SI{6.35}{\mm} it is observed that pertubations increase the count rate by 15\%.

