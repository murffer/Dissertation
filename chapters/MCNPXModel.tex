% MCNPX Model of the interaction rate
\subsection{MCNPX Detector Modeling}
\label{sec:MCNPDetectorModelingMethod}
The performance of films is simulated in MCNPX, a Monte Carlo transport code\cite{pelowitz_mcnpx_????}.
The geometry is as in the PNNL reports, namely a nano-gram of \iso[252]{Cf}  encased in \SI{0.5}{\cm} of lead and \SI{2.5}{\cm} of HDPE. 
The size of the RPM8 is \SI{12.7}{\cm} deep, by \SI{30}{\cm} wide and \SI{2}{\m} tall.
The interaction rate, $I_{\text{sim}}$ provides the total number of simulated neutron interactions in the detector and is calculated using the a cell flux tally in MCNPX and a tally multiplier card.
The reaction rate $\iso[6]{Li}\left(\text{n},\text{t}\right)\alpha$ can be calculated by then applying the appropriate input for the FMn card and using an F4 card to calculate $\phi(E)$.
This is in accordance with the direct evaluation of the PNNL criteria, which require a absolute neutron count rate of \SI{2.5}{count\per\second\per\nano\gram\iso[252]{Cf}}.
Note that in this calculation the source strength is set to be \SI{1}{\nano\gram} \iso[252]{Cf}, which has a neutron emission rate of \SI{2.3E3}{neutron\per\second}.
\begin{align}
  \label{eqn:RPM8InteractionRate}
  I_{\text{sim}} &= S_0 I \\
  &= \SI{2.3E3}{neutron\per\second} I
\end{align}

