\chapter{Executive Summary}
The shortage of \iso[3]{He} has caused the Domestic Nuclear Detection Office (DNDO) within the U.S. Department of Homeland Security (DHS) to sponsor research for alternerative radiation portal monitors (RPM).
The DNDO/DHS in conjunction with Pacific Northwest National Lab (PNNL) have established performance criteria that replacment technologies must satisify.
Replacement technologies must fulfill three basic criteria: 1) a neutron detection efficiency, 2) a gamma insensitivity, and 3) the performance of the detector should not suffer in the the presence of a strong gamma field.
Polymeric films containing \iso[6]{Li} ranging from 15 to 300 microns have the ability to fulfill these criteria if suitably utilized.
For a typical detector this involves maximizing the neutron-gamma discrimination, maximizing the physical detector configuration in order to ensure optimal use of the incident neutron spectra, and ensuring that the scintillation light generated can be collected.

A pulse height discriminator for rejecting gamma interactions has been determined for polymeric films ranging from 15 microns to 300 microns in thickness. 
The ability for this pulse height discriminator has been investigated and attributed to the energy deposition in the material by the reaction products of the gammas and neutrons, where the Compton scattered electrons from a photon interaction generally have energies in the hundreds of keV while the \iso[6]{Li} reaction products have energies in the 10 keV range. 
Detailed GEANT4 simulations indicate that a desired film thickness is around 100 microns. 

A replacement portal monitor has then been designed for layered polymeric films that effectively utilize the \iso[6]{Li} in the detector material. 
These layered detector designs consist of 100 micron, \iso[6]{Li} fluoride loaded polymers that are encased in \SI{5}{\mm} of a wavelength shifting plastic.
A genetic algorthim was employed to optimize the interaction rate above the necesarry discriminator level for three differnet types of detector materials, a polystyrene based film, a polyethtylene napthalate film, and a commerical \iso[6]{LiF}:ZnS(Ag) scintillator (EJ-426).
\begin{table}
  \begin{tabular}{m{3cm} m{5cm} m{3cm}}
    \toprule
    Composition & Film Positions & Expected Photons Colletected \\
    \midrule
    10\% \iso[6]{LiF} Polystyrene &  & 400 \\
    10\% \iso[6]{LiF} Polyepthylene napthalate &  & 500 \\
    \iso[6]{LiF} loaded ZnS:Ag & & \\
    \bottomrule
  \end{tabular}
\end{table}
In the case of the polyethylene napthalate film an alternative method of neutron - gamma discrimination shows promise in which the entire neturon spectra can be utilized. 
In this case only two detector films are necessary, 
Three such layers (10 precent loaded with enriched LiF) can achieve an interaction rate of 3.82 interactions per second per nanogram of Cf-252 (using 12.6 grams of \iso[6]{Li}), while five layer can achieve an interaction rate of 5.31 interactions per second per nanogram of Cf-252 (using 21.0 grams of \iso[6]{Li}) and ten layers an achieve an interaction rate of 7.56 interactions per second per nanogram of Cf-252 (using 41.2 grams of \iso[6]{Li}). 
If two photomultiplier tubes are placed at the top and bottom of a fishtail light guide mounted on the top and bottom of the detector cabinet, eight precent of the optical photons generated in a 10 precent loaded polystyrene film can be collected, leaving 400 photons to produce a signal on the photomultiplier tubes. 

In addition to layered designs a design in which the \iso[6]{Li} detector layers are wrapped around into concentric clyinders was also explored. 
A design using four cylinders (each 2 cm in outer diameter) placed equidistant in the RPM8 loaded with 30\% \iso[6]{Li} fluoride (using \SI{28}{\g} of \iso[6]{Li}) would have an interaction rate above 3.2 interactions per second per nanogram \iso[252]{Cf}.
