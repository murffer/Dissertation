\chapter*{Abstract}
\label{chap:abstract}
Alternative neutron detection technologies are required to replace the current He-3 based Radiation Portal Monitors (RPM) which are employed to detect special nuclear material that may be entering the United States illicitly.
Replacement technologies must fulfill the following criteria established by the Department of Homeland Security: 1) a neutron detection efficiency, 2) a gamma insensitivity, and 3) the performance of the detector should not suffer in the the presence of a strong gamma field.
Several candidate detector designs have been proposed, including boron lined straw tubes, boron trifluoride gas detectors, scintillating Li-6 doped glass detectors, and other Li-6 based scintillators.
Polymeric films containing Li-6 ranging from 15 to 300 microns have the ability to fulfill these criteria if suitably utilized.
For a typical detector material the design involves maximizing the neutron-gamma discrimination, maximizing the physical detector configuration in order to ensure optimal use of the incident neutron spectra, and ensuring that the scintillation light generated can be collected.

A technique for using a pulse height discriminator for rejecting gamma interactions has been determined for polymeric films  ranging from 15 microns to 300 microns in thickness.
The basis of this technique has been attributed to the relative ranges of the secondary electrons of the Compton scattered electrons (generally in the hundred of keV) from the gamma interactions, compared to the ranges of the electrons from the reaction products of a neutron absorption, which have energies int he 10 keV range.
Detailed Monte Carlo simulations indicate that a desired film thickness is around 100 microns.

A replacement portal monitor has been designed for layered polymeric films that effectively utilize the Li-6 in the detector material using a genetic algorithm to optimize the spacing between the layers, while cylindrical designs could also be employed.
If two photomultiplier tubes are placed at the top and bottom of a fishtail light guide mounted on the top and bottom of the detector cabinet, eight precent of the optical photons generated in a 10 precent loaded polystyrene film can be collected, leaving an acceptable number of photons to create a signal.
