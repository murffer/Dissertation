Alternative neutron detection technologies are required to replace the current \iso[3]{He} based Radiation Portal Monitors.
Replacement technologies must fulfill two basic criteria: 1) a neutron detection efficiency, and 2) the performance of the detector should not suffer in the the presence of a strong gamma field.
Polymeric films containing \iso[6]{LiF} ranging from \SI{15}{\um} to \SI{300}{\um} have the ability to fulfill these criteria if suitably utilized.
There exists a need to model and optimize these detector designs in order to ensure that the best performance is reached. 
For a typical detector this involves maximizing the neutron-gamma discrimination, maximizing the physical detector configuration in order to ensure optimal use of the incident neutron spectra, and ensuring that the scintillation light generated can be collected.

It is therefore proposed to use high fidelity energy deposition calculations with the GEANT4 toolkit to model the energy deposition from neutron and gamma events in thin films.
These calculations will guide choosing the optimal detector material thickness in order to maximize the discrimination between neutron and gamma events.
After selecting an appropriate detector material and thickness, it is proposed to utilize a Monte Carlo transport code, MCNPX, to simulate the neutronic interactions of a full scale RPM as it would be tested.
This model will then be optimized to ensure effective use the neutron absorber (\iso[6]{Li}).
Finally, it is proposed to simulate the optimized neutronic model in GEANT4, including light transport to ensure that the designed detector would be a viable alternative in the field.
