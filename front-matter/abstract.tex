\chapter*{Abstract}
\label{chap:abstract}
Alternative neutron detection technologies are required to replace the current He-3 based Radiation Portal Monitors (RPMs).
RPMs are placed at border crossings into the United States in order to detect special nuclear material that may be entering the United States illicitly.
Replacement technologies must fulfill the following criteria established by the Department of Homeland Security: 1) a neutron detection efficiency, 2) a gamma insensitivity, and 3) the performance of the detector should not suffer in the the presence of a strong gamma field.
Several candidate detector designs have been proposed, including boron lined straw tubes, boron trifluoride gas detectors, scintillating Li-6 doped glass detectors, and other Li-6 based scintillators.
Polymeric films containing Li-6 ranging from 15 to 300 microns have the ability to fulfill these criteria if suitably utilized.
There exists a need to model and optimize these detector designs in order to ensure that the best performance is obtained. 
For a typical detector the design involves maximizing the neutron-gamma discrimination, maximizing the physical detector configuration in order to ensure optimal use of the incident neutron spectra, and ensuring that the scintillation light generated can be collected.

A technique for using a pulse height discriminator for rejecting gamma interactions has been determined for polymeric films  ranging from 15 microns to 300 microns in thickness.
The basis of this technique has been attributed to the relative ranges of the secondary electrons of the Compton scattered electrons from the gamma interactions, compared to the ranges of the electrons from the reaction products of a neutron absorption.
The Compton scattered electrons from a photon interaction generally have energies in the hundreds of keV while the Li-6 reaction products have energies in the 10 keV range, which results in the most of the energy from Compton scattered electrons escaping a thin (250 microns or less) film.
Detailed Monte Carlo (GEANT4) simulations indicate that a desired film thickness is around 100 microns.

A replacement portal monitor has then been designed for layered polymeric films that effectively utilize the Li-6 in the detector material.
These layered detector designs consist of 100 micron, Li-6 fluoride loaded polymers that are encased in four millimeters of a wavelength shifter.
A genetic algorithm was used to optimize a layered detector design that is capable of meeting the replacement RPM criteria.
A design in which four cylindrical tubes was also simulated, which is also capable of meeting the replacement RPM criteria while using more Li-6 for the same interaction rate as a layered detector design.
The light collection of the layered detector assemblies was also determined by Monte Carlo simulation.
If two photomultiplier tubes are placed at the top and bottom of a fishtail light guide mounted on the top and bottom of the detector cabinet, eight precent of the optical photons generated in a 10 precent loaded polystyrene film can be collected, leaving an acceptable number of photons to create a signal.
