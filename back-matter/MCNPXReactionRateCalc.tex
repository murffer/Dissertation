% MCNPX Model of the interaction rate
\chapter{MCNPX Tally Interactions}
The performance of films is simulated in MCNPX, a Monte Carlo transport code\cite{pelowitz_mcnpx_2006}.

The interaction rate is calculated using the a cell flux tally in MCNPX and a tally multiplier card.
The tally multiplier card (FMn) is used to calculated any quantity of the form \eqref{eqn:FMCardForm} \cite{pelowitz_mcnpx_2006}
\begin{align}
  \label{eqn:FMCardForm}
  I &= C\int\phi(E)\Re_m(E)dE
\end{align}
where \definevar{$I$}{Interaction rate}, \definevar{$\phi(E)$}{Energy dependent fluence} , \definevar{$\Re_m(E)$}{Response function operator} and $C$ is an arbitrary scalar for normalization.
An general example of the use of the FM card is shown in Listing \ref{lst:GeneralFMExample}, which is taken from the MCNPX manual \cite{pelowitz_mcnpx_2006}.
% See pg. 4-41 of the MCNP manual
\begin{lstlisting}[caption={[Example usage of the FM card]Example usage of the FM card to calculate the number of reactions per \si{\cm\cubed} of type R in cell 8 of material M. The normalization is by atomic density, signified by the -1},label={lst:GeneralFMExample}]
F104: N 8
FM104 -1 M R
\end{lstlisting}

The reaction rate $\iso[6]{Li}\left(\text{n},\text{t}\right)\alpha$ can be calculated by then applying the appropriate input for the FMn card and using an F4 card to calculate $\phi(E)$.
It should be noted that depending on the form of the cell flux card it may be necessary to normalize by the volume of the cell, $\forall$.
\nomenclature{$\forall$}{Volume of the cell}

This is shown in Listing \ref{lst:InteractionRateRPM}, where the reaction number is 105 and the material number of the detector is 3.
The interaction rate in a simulated RPM8 replacement detector is calculated in a similar manner as the simulation of the measured detectors; the interaction rate as computed by the \verb+FMn+ is multiplied by the source strength and volume if necessary.
An example of the MCNPX input cards is shown in Listing \ref{lst:InteractionRateRPM}.
Given that there the thermal response is not desired, there is no need to subtract out the differences between the spectra, and the interaction rate is simply \eqref{eqn:RPM8InteractionRate}.
Note that in this calculation the source strength is set to be \SI{1}{\nano\gram} \iso[252]{Cf}, which has a neutron emission rate of \SI{2.3E3}{neutron\per\second}.
This is in accordance with the direct evaluation of the PNNL criteria, which require a absolute neutron count rate of \SI{2.5}{count\per\second\per\nano\gram\iso[252]{Cf}}.
\begin{lstlisting}[caption={[RPM8 ${}^{6}\text{Li}\left(\text{n},\text{t}\right)\alpha$ Reaction Rate]RPM8 ${}^{6}\text{Li}\left(\text{n},\text{t}\right)\alpha$ Reaction Rate. The detector is all of the layers of cell 500 inside universe 610. This tally is multiplied by an SD card to normalize by the volume},label={lst:InteractionRateRPM}]
FC4 (n,t) Reactions in Thin Film (Neutron Detector)
F4:n (500<610)
SD4 1
FM4 -1 3 105
\end{lstlisting}
\begin{align}
  \label{eqn:RPM8InteractionRate}
  I_{\text{sim}} &= S_0 I \\
  &= \SI{2.3E3}{neutron\per\second} I
\end{align}

$I_{\text{sim}}$ provides the total number of simulated neutron interactions in the detector.

\section{Example of Layered Detector Geometry}
The following tables provide examples of the positions of the layered geometry used for the genetic algorithm. 
These geometries correspond to \autoref{fig:RPMLayeredRendering25} and \autoref{fig:RPMLayeredRendering5}, and \autoref{fig:RPMLayeredRendering75}.
% Optimal geometry tables.  All are PS Films
\begin{table}
	\caption[Optimal Layered Film Geometry for 7.5 interaction per second per nanogram Cf-252]{Optimal layered film geometry for an interaction rate of 7.5 interactions per second per nano-gram \iso[252]{Cf} in a 10\% \iso[6]{Li} loaded PS film. The positions shown in the table are the right boundary, where the detector starts at \SI{0.0}{\cm}. The genome representing this geoemtry is \texttt{01111101110100001000}.}
	\begin{tabular}{m{3cm} m{4cm}}
	\toprule
	Edge Position (\si{\cm}) & Material \\
	\midrule
0.635&Moderator\\
0.645&Detector\\
1.270&LightGuide\\
1.280&Detector\\
1.905&LightGuide\\
1.915&Detector\\
2.540&LightGuide\\
2.550&Detector\\
3.175&LightGuide\\
3.185&Detector\\
3.810&LightGuide\\
4.445&Moderator\\
4.455&Detector\\
5.080&LightGuide\\
5.090&Detector\\
5.715&LightGuide\\
5.725&Detector\\
6.350&LightGuide\\
6.985&Moderator\\
6.995&Detector\\
7.620&LightGuide\\
8.255&Moderator\\
8.890&Moderator\\
9.525&Moderator\\
10.160&Moderator\\
10.170&Detector\\
10.795&LightGuide\\
11.430&Moderator\\
12.065&Moderator\\
12.700&Moderator\\
	\bottomrule
	\end{tabular}
\end{table}
\begin{table}
	\caption[Optimal Layered Film Geometry for 5.0 interaction per second per nanogram Cf-252]{Optimal layered film geometry for an interaction rate of 5.0 interactions per second per nano-gram \iso[252]{Cf}  in a 10\% \iso[6]{Li} loaded PS film. The positions shown in the table are the right boundary, where the detector starts at \SI{0.0}{\cm}. The genome representing this geoemtry is \texttt{011101001000000}.}
	\begin{tabular}{m{3cm} m{4cm}}
	\toprule
	Edge Position (\si{\cm}) & Material \\
	\midrule
0.847&Moderator\\
0.857&Detector\\
1.693&LightGuide\\
1.703&Detector\\
2.540&LightGuide\\
2.550&Detector\\
3.387&LightGuide\\
4.233&Moderator\\
4.243&Detector\\
5.080&LightGuide\\
5.927&Moderator\\
6.773&Moderator\\
6.783&Detector\\
7.620&LightGuide\\
8.467&Moderator\\
9.313&Moderator\\
10.160&Moderator\\
11.007&Moderator\\
11.853&Moderator\\
12.700&Moderator\\
	\bottomrule
	\end{tabular}
\end{table}
\begin{table}
	\caption[Optimal Layered Film Geometry for 2.5 interaction per second per nanogram Cf-252]{Optimal layered film geometry for an interaction rate of 2.5 interactions per second per nano-gram \iso[252]{Cf} in a 10\% \iso[6]{Li} loaded PS film. The positions shown in the table are the right boundary, where the detector starts at \SI{0.0}{\cm}. The genome representing this geoemtry is \texttt{0011010000}.}
	\begin{tabular}{m{3cm} m{4cm}}
	\toprule
	Edge Position (\si{\cm}) & Material \\
	\midrule
1.270&Moderator\\
2.540&Moderator\\
2.550&Detector\\
3.810&LightGuide\\
3.820&Detector\\
5.080&LightGuide\\
6.350&Moderator\\
6.360&Detector\\
7.620&LightGuide\\
8.890&Moderator\\
10.160&Moderator\\
11.430&Moderator\\
12.700&Moderator\\
	\bottomrule
	\end{tabular}
\end{table}
