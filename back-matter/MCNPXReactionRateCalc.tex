% MCNPX Model of the interaction rate
\chapter{MCNPX Tally Interactions}
The performance of films is simulated in MCNPX, a Monte Carlo transport code\cite{pelowitz_mcnpx_????}.

The interaction rate is calculated using the a cell flux tally in MCNPX and a tally multiplier card.
The tally multiplier card (FMn) is used to calculated any quantity of the form \eqref{eqn:FMCardForm} \cite{pelowitz_mcnpx_2006}
\begin{align}
  \label{eqn:FMCardForm}
  I &= C\int\phi(E)\Re_m(E)dE
\end{align}
where \definevar{$I$}{Interaction rate}, \definevar{$\phi(E)$}{Energy dependent fluence} , \definevar{$\Re_m(E)$}{Response function operator} and $C$ is an arbitrary scalar for normalization.
An general example of the use of the FM card is shown in Listing \ref{lst:GeneralFMExample}, which is taken from the MCNPX manual \cite{pelowitz_mcnpx_????}.
% See pg. 4-41 of the MCNP manual
\begin{lstlisting}[caption={[Example usage of the FM card]Example usage of the FM card to calculate the number of reactions per \si{\cm\cubed} of type R in cell 8 of material M. The normalization is by atomic density, signified by the -1},label={lst:GeneralFMExample}]
F104: N 8
FM104 -1 M R
\end{lstlisting}

The reaction rate $\iso[6]{Li}\left(\text{n},\text{t}\right)\alpha$ can be calculated by then applying the appropriate input for the FMn card and using an F4 card to calculate $\phi(E)$.
It should be noted that depending on the form of the cell flux card it may be necessary to normalize by the volume of the cell, $\forall$.
\nomenclature{$\forall$}{Volume of the cell}

This is shown in Listing \ref{lst:InteractionRateRPM}, where the reaction number is 105 and the material number of the detector is 3.
The interaction rate in a simulated RPM8 replacement detector is calculated in a similar manner as the simulation of the measured detectors; the interaction rate as computed by the \verb+FMn+ is multiplied by the source strength and volume if necessary.
An example of the MCNPX input cards is shown in Listing \ref{lst:InteractionRateRPM}.
Given that there the thermal response is not desired, there is no need to subtract out the differences between the spectra, and the interaction rate is simply \eqref{eqn:RPM8InteractionRate}.
Note that in this calculation the source strength is set to be \SI{1}{\nano\gram} \iso[252]{Cf}, which has a neutron emission rate of \SI{2.3E3}{neutron\per\second}.
This is in accordance with the direct evaluation of the PNNL criteria, which require a absolute neutron count rate of \SI{2.5}{count\per\second\per\nano\gram\iso[252]{Cf}}.
\begin{lstlisting}[caption={[RPM8 ${}^{6}\text{Li}\left(\text{n},\text{t}\right)\alpha$ Reaction Rate]RPM8 ${}^{6}\text{Li}\left(\text{n},\text{t}\right)\alpha$ Reaction Rate. The detector is all of the layers of cell 500 inside universe 610. This tally is multiplied by an SD card to normalize by the volume},label={lst:InteractionRateRPM}]
FC4 (n,t) Reactions in Thin Film (Neutron Detector)
F4:n (500<610)
SD4 1
FM4 -1 3 105
\end{lstlisting}
\begin{align}
  \label{eqn:RPM8InteractionRate}
  I_{\text{sim}} &= S_0 I \\
  &= \SI{2.3E3}{neutron\per\second} I
\end{align}

$I_{\text{sim}}$ provides the total number of simulated neutron interactions in the detector.
