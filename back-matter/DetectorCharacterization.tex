\section{Detector Characterization}
Repeated characterization techniques of the detector materials is necessary to ensure a fair comparison between differnet detector materials.
As the focus of this work was on effective scintiallators, the characteizations were designed to measure scintillation properties; namely the light yield of the detector and the count rate when exposed to differnet radiation sources.

\subsection{Characterization Sources}

\subsection{Light Yeild}
The light yeild of a spectrum describes how many photons are generated (and subsquently detected on a PMT) for a given material for a scintillation event from a radiation source.
% Some words from Knoll about LY

The light yield of fabricated samples that were characterized was completed by comparing the spectrum average of a sample to that a sample of a known light yield.
Formally this is shown in \eqref{eqn:LightYield} where \definevar{p(x)}{measured spectrum}, 
This is described by the Birks equation \eqref{eqn:BirksEquation}
\begin{align}
  \label{eqn:BirksEquation}
  \frac{dL}{dx} = \frac{S_B\frac{dE}{dx}}{1+kB\frac{dE}{dx}}
\end{align}
where \definevar{$S_B$}{absolute scintillation efficiency},\definevar{$\frac{dE}{dx}$}{linear stopping power} and \definevar{$kB$}{Birks parameter}.

\subsection{Count Rate}
