\chapter{Detector Characterization}
\label{chap:DetChar}
Repeated characterization techniques of the detector materials are necessary to ensure a fair comparison between differnet detector materials.
As the focus of this work was on effective scintiallators, the characteizations were designed to measure scintillation properties; namely the light yield of the detector and the count rate when exposed to differnet radiation sources.

\subsection{Characterization Electronics and Sources}
Solid samples are characterized by mounting them with a thin layer of silicone optical grease (BC-630, index of refraction 2.465) onto a Philips XP2202B 10 Stage PMT most senstive in the \SI{350}{\nm} to \SI{500}{\nm} region.
The PMT is then connected to a Canberra 2007P base, which also funtions as a preamplifier.
The Canberra 2007P feeds into an Ortec 572A amplifier, and the amplified signal is inputted to an Ortec 926 MCB-ADC.
MAESTRO-32 is the used to read the signals from the MCB.
\autofref{fig:ElectronicsSetup} provides an overview of this setup.
\begin{figure}
  \includegraphcis[width=0.5\textwidth]{}
  \capition[Optical Characterization Experiment Setup]{Electronic figure for measuring the spectra of materials in response to various radiation sources.}
  \label{fig:ElectronicSetup}
\end{figure}
A general protocol has been developed in order to ensure that the measurments are made in a repeatable manner and verified with a reference.
\begin{enumerate}
\end{enumerate}
\subsection{Light Yeild}
The light yeild of a spectrum describes how many photons are generated (and subsquently detected on a PMT) for a given material for a scintillation event from a radiation source.
% Some words from Knoll about LY

The light yield of fabricated samples that were characterized was completed by comparing the spectrum average of a sample to that a sample of a known light yield.
Formally this is shown in \eqref{eqn:LightYield} where \definevar{p(x)}{measured spectrum}, 

\subsection{Count Rate}
