% XSDRN Model of the RPM
The XSDRN model was a simplified model of the RPM along an axis through the midpoint of the RPM.
A $S_n$ of 16 was used for the quadrature, and convergence for the flux was set at \SI{1E-7} for the inner iterations.
Only two types of materials were simulated in the XSDRN calculation; a detector material containing the \iso[6]{LiF} and a moderating material of polystyrene.
A 44 group neutron cross sections of each of these materials were processed using NITWAL \cite{NITAWL_2011} (without any resonance regions) assuming an infinite, homogeneous medium for simplicity.
The XSDRN model consisted of a multi-group isotropic boundary source on the left most boundary on the RPM, with the values for this flux calculated by an MCNPX simulation.
A MCNPX calculation was used to determine the neutron flux incident upon the left most side of the RPM, and then this flux was input as the surface boundary flux condition in XSDRN.
The number of neutron absorptions was calculated using an activity flag in the XSDRN model.
The fitness function for this model was implemented as the activity normalized the number of detector layers.
The number of layers was chosen as the normalization factor instead of the actual mass of absorber to allow for different film compositions to be simulated more readily.

The comparison between the MCNPX simulation and the XSDRN is shown for some of the samples in \autoref{tab:10GenomeXSDRNMCNPXCompare} and \autoref{tab:20GenomeXSDRNMCNPXCompare}, where the change in rank is computed by rank of the MCNPX model versus the rank of the XSDRN model.
It is observed that the XSDRN model preformed fairly closely to the MCNPX model, but tended to over predict and favor geometries that had repeated layers and clusters.
This was very noticeable when the geometries started with a neutron absorber layer this is attributed to the breakdown of the diffusion equation in a strongly absorbing medium near a source.
Some stratification of the results were also observed, leading to the conclusion that the XSDRN calculations should only be used as a general guide.
More details are available in the simulation code base, summarized in \autoref{sec:garpm8opt}.
\begin{table}
  \caption[10 Genome Length RPM Model]{10 Genome Length RPM Model Interactions rates}
  \label{tab:10GenomeXSDRNMCNPXCompare}
  \begin{tabular}{c c | c c | c}
    \toprule
    Genome & Activity & Interaction Rate & Rank Change \\
    \midrule
  0011010000& 9.30 &  3.82 & $\downarrow$ 13 \\
  0110100000 & 10.50  &  3.81 & 0 \\
  0101010000 & 10.12  & 3.79 & $\downarrow$ 7 \\
 0101100000 &  & 3.79 & $\downarrow$ 1\\
  0011100000 & 9.63 &  3.77 & $\downarrow$ 3 \\
    \bottomrule
  \end{tabular}
\end{table}
\begin{table}
  \caption[20 Genome Length RPM Model]{20 Genome Length RPM Model Interactions rates}
  \label{tab:20GenomeXSDRNMCNPXCompare}
  \begin{tabular}{c c | c c | c}
    \toprule
    Genome & Activity  & Interaction Rate & Rank Change \\
    \midrule
  00100101000000000000 & 7.77 & 3.79 & $\downarrow$ 19 \\
  00011000100000000000 &  & 3.78 &  \\
  00011000010000000000 &  &  3.76 &  \\
  00110001000000000000 &  &  3.69 & $\downarrow$ 15\\
  01011010010000000000 & 23.46 & 3.66 & $\uparrow$ 1\\
    \bottomrule
  \end{tabular}
\end{table}
