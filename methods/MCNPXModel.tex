% MCNPX Model of the interaction rate
The performance of films is simulated in MCNPX, a Monte Carlo transport code\cite{pelowitz_mcnpx_2006}.
The geometry is as in the PNNL reports, namely a nano-gram of \iso[252]{Cf}  encased in \SI{0.5}{\cm} of lead and \SI{2.5}{\cm} of HDPE. 
The size of the RPM8 is \SI{12.7}{\cm} deep, by \SI{30}{\cm} wide and \SI{2}{\m} tall.
The interaction rate, $I_{\text{sim}}$ provides the total number of simulated neutron interactions in the detector and is calculated using the a cell flux tally in MCNPX and a tally multiplier card.
The reaction rate $\iso[6]{Li}\left(\text{n},\text{t}\right)\alpha$ can be calculated by then applying the appropriate input for the FMn card and using an F4 card to calculate $\phi(E)$.
This is in accordance with the direct evaluation of the PNNL criteria, which require a absolute neutron count rate of \SI{2.5}{count\per\second\per\nano\gram\iso[252]{Cf}}.
Note that in this calculation the source strength is set to be \SI{1}{\nano\gram} \iso[252]{Cf}, which has a neutron emission rate of \SI{2.3E3}{neutron\per\second}.
\begin{align}
  \label{eqn:RPM8InteractionRate}
  I_{\text{sim}} &= S_0 I \\
  &= \SI{2.3E3}{neutron\per\second} I
\end{align}
However, not all of these interactions will lead to counts above the pulse height discriminator setting necessary for meeting the gamma intrinsic efficiency.
This is corrected for by scaling $I_{\text{sim}}$ by the fraction of counts, $\eta$, that occur above the gamma LLD \eqref{eqn:FractionOfCountsDefination}, \eqref{eqn:RPMCountRate}.
\begin{align}
  \label{eqn:FractionOfCountsDefination}
  \eta \equiv \frac{\int_{MLLD}^\infty p(x)dx}{\int_0^\infty p(x)dx}
\end{align}
\begin{align}
 \label{eqn:RPMCountRate}
 \text{Count Rate} &= I_{\text{sim}} \eta
\end{align}
 
The MCNPX model was benched marked for characterized samples in the in house \iso[252]{Cf} neutron irradiator by simulating the interaction rate and comparing it to the measured count rate.
These results are summarized in \autoref{tab:MCNPXVal}.
It should be noted that the interaction rate is not directly the count rate as an interaction needs to have its scintillation photons collected in order to be a count. 
One would then expect the simulated interaction rate to be a few precent higher than the measured count rate.
In addition the polymeric sample composition is determined before pressing or casting providing a means for the uncertainty of the composition.
The relative error is defined as $(sim-obs)/obs)$.
\begin{table}
	\centering
	\caption[MCNPX Neutron Validation Results]{Comparison between simulated neutron interaction rate and measured count rate. It is expected that some of the uncertainity in the fabricated samples comes from not knowing exactly the amount of \iso[6]{Li} in the film, as it is determined before casting or pressing.}
	\label{tab:MCNPXVal}
	\begin{tabular}{m{4cm} | p{3cm} p{3cm} p{2cm}}
		\toprule
			Sample & Simulated Interaction Rate & Measured Count Rate & Relative Error \\
		\midrule
			GS20 & 424  & 428 & 0.7\%  \\ 
			PS 30\% LiF \SI{50}{\um} &  56 & 51 & 9.5\% \\
			PS 30\% LiF \SI{25}{\um} & 108 & 96 &13\%  \\
			PEN, 10\% LiF, \SI{110}{\um} & 75.1 & 70 & 7\% \\
			EJ426 HD2 & 226 & 224 & 0.8\% \\
		\bottomrule
	\end{tabular}
\end{table}

The gamma irradiator fluence was calibrated by measuring the dose rate are various positions, and then simulating the dose rate in MCNPX.
The results of this study are shown in \autoref{tab:MCNPXPhotonFluxVal}, and in general good agreement is observed between the simulation and the measurement.
In some cases where the simulated dose rate is higher may be due to uncertainty in the exact location of the probe in the measurement.
\begin{table}
	\centering
	\caption[MCNPX Photon Dose Rate Validation Results]{Comparison between simulated dose rate and measured dose rate.}
	\label{tab:MCNPXPhotonFluxVal}
	\begin{tabular}{c  c |c  c}
		\toprule
		\multicolumn{2}{c}{Measured} & \multicolumn{2}{c}{Simulated} \\
		Distance  & Dose Rate & Distance & Dose Rate \\
		\midrule
		\SI{10.2}{\cm} & \SI{10}{mRem \per h} & \SI{10.2}{\cm} & \SI{10.3}{mRem \per h} \\
		\SI{13}{\cm} & \SI{5.5}{mRem \per h} & \SI{12.8}{\cm} & \SI{5.4}{mRem \per h} \\
		\SI{28}{\cm} & \SI{2}{mRem \per h} & \SI{28}{\cm} & \SI{1.8}{mRem \per h} \\
		\bottomrule
	\end{tabular}
\end{table}
