One would like to collect as much light as emitted from the scintillator as possible.
In practice two effects limit the fraction of the emitted light collected; the optical self-absorption in the material and losses at the edges of the optical surfaces \cite{knoll_radiation_2009}.
The optical self-absorption of a scintillator is a material property in which photons are reabsorbed by the material.
This effect is important for large area scintillators and for scintillators which are not optically  clear, both of which apply to the developed polymeric scintillators.
Typically this effect is mitigated by the use of a wavelength shifting fiber in which the light is transferred to a material which has a much lower optical self-absorption.

Light collection of a scintillation event is emitted isotropically, and therefore only a very small fraction of the photons can travel directly to a photon detector surface.
The majority of the light must then be collected by reflecting back into the medium.
Snell's law governs the reflection of light at an optical boundary, and there are two cases to consider as shown in \autoref{fig:SnellsLaw} and described by Snell's Law, \eqref{eqn:SnellsLaw}
\begin{align}
	\theta_c = \sin^-1 \frac{n_1}{n_0}
	\label{eqn:SnellsLaw}
\end{align}
where \definevar{$\theta_c$}{critical angle}, \definevar{$n_1$}{index of refraction of the surrounding medium} and \definevar{$n_0$}{index of refraction of the scintillator}.
If the angle of incidence, $\theta$, is greater than the critical angle total internal reflection will occur.
When the angle of incidence is less than the critical angle particle reflection (or \textit{Fresnel} reflection) will occur and there will be partial transmission of the photons to the surrounding medium.
\begin{figure}
	\centering
	\includegraphics[width=\textwidth]{SnellsLaw}
	\caption[Light Reflection at a Boundary]{Relfection of light at an optical surface is governed by Snell's law.  The fraction of light reflected back into the material is greatest at an angle of incident equal to $\theta_c$}
	\label{fig:SnellsLaw}
\end{figure}
To ensure that the light stays within the desired medium it is usually encased in a reflector, of which there are two types (\autoref{fig:SpecularDiffusive}).
A polished metallic surface (such as aluminized mylar) may be applied as a specular reflector which are generally better when the length is much longer than the thickness\cite{SaintGobain_DAM_2012}.
A diffusive reflector, such as teflon tape, is better for conditions when the detector is thick compared to its length \cite{knoll_radiation_2009}.
\begin{figure}
	\centering
	\includegraphics[width=\textwidth]{SpecularDiffusive}
	\caption[Specular and Diffusive Reflection]{Specular reflection (right) in which the light in a single incoming direction is emitted as a single outgoing direction. The rough surface of a diffusive reflector, left, causes the light to be reflected at many angles.}
	\label{fig:SpecularDiffusive} 
\end{figure}
It should be noted, however, that one would like to optically match the surface at which the scintillator is to be viewed to prevent reflection.

In the cases of a large scintillating detector, such as the one presented in this work, it  may be necessary to employ more than one PMT to collect the light.
In such cases the use of light pipes may enhance the collection efficiency. 
Light pipes are not without costs, however, as they are generally of a high index of refraction to maintain a high internal reflection
