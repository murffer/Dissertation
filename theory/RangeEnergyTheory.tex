%%%%%%%%%%%%%%%%%%%%%%%%%%%%%%%%%%%%%%%%%%%%%%%%%%%%%%%%%%%%%%%%%%%%%%%%%%%
%                                                                         %
%                      ENERGY SCALE AND RANGES                            %
%                                                                         %
%%%%%%%%%%%%%%%%%%%%%%%%%%%%%%%%%%%%%%%%%%%%%%%%%%%%%%%%%%%%%%%%%%%%%%%%%%%
The theoretical basis for the difference in energy deposition lies in the different mechanisms between charged particle interactions and photon interactions in matter.
For photons with energies between approximately \SI{0.5}{\MeV} and \SI{5}{\MeV} Compton scattering is the predominant interaction mechanism between the material and the photon.
The probability of an electron having a given kinetic energy after scattering can be expressed as \autoref{eqn:dSdEKleinNishina} (see \autoref{chap:ComptonScatter} for derivation details)
\begin{align}
  \label{eqn:dSdEKleinNishina}
\frac{d\sigma}{dE_e} = 2\pi r_e^2 \sin \theta f(\theta)\left [ \frac{1+\frac{E}{m_e c^2}\left(1-\cos\theta \right)^2}{E^2 \sin \theta} \right ]
\end{align}
provided  $f(\theta)$ is defined as $f(\theta) = \frac{1}{2}\left(\frac{E'}{E}\right)^2 \left(\frac{E'}{E} + \frac{E}{E'}-\sin^2\theta\right)$.
This distribution is shown for \iso[60]{Co} in \autoref{fig:Co60ComptonScatterSpectra}.
It is then highly likely that the Compton scattered electron will have an energy close to that of the maximum energy for the Compton scattering, \SI{0.96}{\MeV} for the \SI{1.173}{\MeV} photon and \SI{1.12}{\MeV} for the \SI{1.33}{\MeV} photon.
\begin{figure}
  \centering
    \includegraphics[width=\textwidth]{Co60ComptonScatteredSpectra}
    \caption[Analytical Co-60 Compton Electron Differential Microscopic Cross Section]{Co-60 Compton Electron Differential Microscopic Cross Section. The relative probabilities of an Compton Scattered Electron having a given kinetic energy from \iso[60]{Co} can be determined by their relative differential microscopic cross sections. For example, it almost two and a half times more likely that a Compton scattered electron will have a kinetic energy of \SI{0.95}{\MeV} than \SI{0.75}{\MeV}.}
    \label{fig:Co60ComptonScatterSpectra}
  \end{figure}
\begin{figure}
    \includegraphics[width=\textwidth]{Co60ComptonScatteredSpectraCDF}
    \caption[Analytical Co-60 Compton Electron Kinetic Energy Cumulative Distribution]{Probability of the energy of a Compton Scattered Electron from \iso[60]{Co}. Over half of the electrons from the Compton scattering will have an energy greater than \SI{0.5}{\MeV}}
    \label{fig:Co60ComptonScatterCDF}
\end{figure}
The cumulative distribution function (CDF) is shown in \autoref{fig:Co60ComptonScatterCDF}.
In the case of $\iso[6]{Li}\left(n,\iso[3]{H}\right)\alpha$ the fission energy is distributed between a triton of energy \SI{2.73}{\mega\eV} and an alpha of energy \SI{2.05}{\mega\eV}.
The maximum kinetic energy of an electron from a Compton scattering event with an impingement \iso[60]{Co} source is \SI{1.117}{\mega\eV} (for the \SI{1.332}{\mega\eV} gamma). 
In polystyrene with a density of \SI{1}{\gram\per\cm\cubed}, the range of the maximum electron from Compton scattering is around \SI{4.5E3}{\um}\cite{berger_estar_2005}.
If elastic scattering between the alpha, triton and electrons is assumed the maximum kinetic energy of an electron is \SI{1.097}{\kilo\eV} for the alpha particle and \SI{1.986}{\kilo\eV} for the triton\cite{turner_atoms_2008}.
The range of the electron from a gamma interaction is more than \num{1E3} times greater than the range of electrons from an alpha or triton (Table \ref{tab:BasicEDepOutline}).
Therefore, it is more likely that the electrons generated by the alpha and the triton deposit significantly more of their energy in a thin film than the electron from a gamma.
This is also reflected in the stopping power, where the reaction product secondary electrons have a stopping power 40 times that of the secondary electrons from a gamma \cite{berger_estar_2005}.
The simulated electron range distributions for several energies is shown in \autoref{fig:ElectronRangesDist} and summarized in \autoref{fig:ElectronRanges}.
The electrons  from Compton scattering of \iso[60]{Co} will be on the order of  \~ 100 keV will have ranges two orders of magnitude than that of electrons from the charged particle interactions.
\begin{table}[ht]
  \caption[Electron Energy, Range, and Stopping Power]{Electron Energy, Range, and Stopping Power \protect\cite{berger_estar_2005,turner_atoms_2008}}
	\centering
	\begin{tabular}{c | c S c}
	\toprule
	{Electron Parent} & {Electron Energy} & {Total Stopping Power} & {CSDA Range} \\
	 &  & \si{\mega\eV \cm\squared \per \gram} & \si{\gram\per\cm\squared} \\
	\midrule
	{gamma}  & \SI{1.12}{\mega\eV} & 1.79 & $\ge$~\num{4.48E-1} \\
	{triton} & \SI{1.99}{\kilo\eV} & 75.1 & $\le$~\num{2.55E-4} \\
	{alpha}  & \SI{1.10}{\kilo\eV} & 113  & $\le$~\num{2.55E-4} \\
	\bottomrule
	\end{tabular}
  \label{tab:BasicEDepOutline}
	% See pg. 87 of Matthew's lab notebook for the calculation
\end{table}
\begin{figure}
  \centering
      \includegraphics[width=\textwidth]{ElectronRangeDistribution}
      \caption[Simulated Electron Ranges Distrubtions in Polystyrene]{Simulated range of \SI{1}{\MeV}, \SI{100}{\keV}, and \SI{10}{\keV} electrons.}
      \label{fig:ElectronRangesDist}
\end{figure}
\begin{figure}
	\centering
      \includegraphics[width=\textwidth]{ElectronRange}
  \caption[Simulated Electron Ranges in Polystyrene]{GEANT4 simulated electron ranges.} 
  \label{fig:ElectronRanges}
\end{figure}

The development of the secondary electron energy ranges so far has been from a simplified treatment of the problem when only the dominant process are considered.
Detailed simulations are necessary in order to have a greater understanding of the problem. 
The energy of the electrons created from an alpha (\SI{2.05}{\MeV}), triton (\SI{2.73}{\MeV}), and gammas from \iso[60]{Co} were calculated using a GEANT4 simulation, and overlaid on the range of electrons (left axis) as shown in \autoref{fig:ERangeAndDist}.
It is then clear that the electrons from the \iso[60]{Co} will travel much father than those from the $\iso[6]{Li}\left(n,\alpha\right)\iso[3]{H}$ reaction products.
\begin{figure}
  \centering
  \includegraphics[width=\textwidth]{ElectronRangeAndEnergyDist}
  \caption[Electron Range and Energy Distribution of Selected Reactions]{The electron range (left axis) overlaid with the kinetic energy of electrons from \SI{2.05}{\MeV} alpha, \SI{2.73}{\MeV} triton, and gammas from \iso[60]{Co}. The \iso[60]{Co} electrons have energies much greater than the alpha and triton}
  \label{fig:ERangeAndDist}
\end{figure}
