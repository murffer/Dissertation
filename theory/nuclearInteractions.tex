Generally neutrons are born at energies on the order of a few MeV, which is several orders of magnitude above thermal energies.
These neutrons than interact with other nuclei in the material.

The probability of a nuclear reaction \eqref{eqn:microXS} can be expressed as probable reaction rate (commonly referred to as the microscopic cross section, $\sigma$) for $n$ neutrons traveling with velocity $v$ a distance $dx$ in a material with an atomic density of $N$.
This also gives rise to the macroscopic cross section, $\Sigma=N\sigma$, where the interpretation is the probability per unit path length of the process described by the microscopic cross section.
\begin{align}
	\label{eqn:microXS}
	\sigma \equiv \frac{\text{reaction rate}{nvNdx}
\end{align}
If the total cross section is known, the mean free path of the neturon can be calculated as $1/\Sigma_\text{tot}$.
In polyethylene the mean free path of a thermal neutron is about \SI{3.7}{\mm}, and will drecrease with the addition of a strongly absorbing material.

In a scattering medium the n, losing energy, until they are eventually arrive at thermal energies. 
In pure hydrogen it takes 27 elastic collisions for a \SI{2}{\MeV} neutron to slow down to \SI{0.025}{\eV}, and \SI{119} collisions in carbon.

In addition to scattering, there are other processes that a neutron can participate in, the most relevant for this work being absorption.

The rate of flow of neutrons is described by the neutron flux and is calculated as the neutron density multiplied by the neutron velocity.
The neutron fluence is then the integrated neutron flux over a given rate of time.

They are several nuclear interactions that are of interest for radiation portal monitors. These reactions are